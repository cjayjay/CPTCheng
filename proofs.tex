

%%%%%%%%%%%%%%%%%%%%%%%%%%%%%%%%%%%%%%%%%%%%%%%%%%%%%%%%%%%%%%%%%%%%%%%%%%%%%%%%
%%%%%%%%%%%%%%%%%%%%%%%%%%%%%%%%%%%%%%%%%%%%%%%%%%%%%%%%%%%%%%%%%%%%%%%%%%%%%%%%
\subsection{Proofs for CPT-value estimator}
\label{appendix:cpt-est}

\subsubsection{Lipschitz continuous weights}
In order to prove Proposition \ref{thm:asymp-conv}, we require the dominated convergence theorem in its generalized form, which is provided below.
\begin{theorem}{\textbf{\textit{(Generalized Dominated Convergence theorem)}}}
Let $\{f_n\}_{n=1}^\infty$ be a sequence of measurable functions on $E$ that converge pointwise a.e. on a measurable space $E$ to $f$.  Suppose there is a sequence $\{g_n\}$ of integrable functions on $E$ that converge pointwise a.e. on $E$ to $g$ such that $|f_n| \leq g_n$ for all $n \in \mathbb{N}$.  
If $\lim\limits_{n \rightarrow \infty}$ $\int_E$ $g_n$ = $\int_E$ $g$, then $\lim\limits_{n \rightarrow \infty}$ $\int_E$ $f_n$ = $\int_E$ $f$.
\end{theorem}


\begin{proof}
This is a standard result that can be found in any textbook on measure theory. For instance, see Theorem 2.3.11 in \cite{athreya2006measure}.
\end{proof}

In addition to the notational convention in section 2, we shall henceforth denote $u^+(X)$ and $u^-(X)$ by $U^+$ and $U^-$, respectively, for notational convenience. 
in the whole section 6.1.
\subsection*{Proof of Proposition \ref{thm:asymp-conv}: Asymptotic convergence}
 
\begin{proof}
Recall that the CPT-value for any r.v. X is defined as
\begin{align*}
V(X) = \intinfinity w^+(P(U^+>x) dx - \intinfinity w^-(P(U^->x) dx. 
\end{align*} 

Also, recall that we estimate $V(X)$ using the empirical distribution as follows:
\begin{align}
\label{eq:cpt-est-appendix}
\widehat V_n(X) = \intinfinity w^+(1-{\hat F_n}^+(x))  dx - \intinfinity w^-(1-{\hat F_n}^-(x))  dx,
\end{align}
where 
$$
{\hat F_n}^+(x)=\frac{1}{n} \sum_{i=1}^n 1_{(U^+ \leq x)}, \quad\text{and}\quad {\hat F_n}^-(x)=\frac{1}{n} \sum_{i=1}^n 1_{(U^- \leq x)}.
$$

We first prove the claim for the first integral in \eqref{eq:cpt-est-appendix}, i.e., we show
\begin{align}
\intinfinity w^+(1-{\hat F_n}^+(x))  dx \rightarrow \intinfinity w^+(P(U^+>x) dx.\label{eq:3}
\end{align} 

Since $w^+$ is Lipschitz continuous with constant $L$, we have almost surely that
$w^{+}(1-\hat{F_n}(x)) \leq L (1-\hat{F_n}(x))$,  
for all $n$ and 
 $w^{+}((P(U^+>x)) \leq L\cdot (P(U^+>x)$, since $w^+(0)=0$.
 
Notice that the empirical distribution function 
${\hat F_n}^+(x)$ 
generates a Stieltjes measure which takes mass 
$1/n$ on each of the sample points $U^+_i$. 

We have
$$\intinfinity (P(U^+>x))  dx = E(U^+)$$
and

\begin{equation}
\intinfinity (1-{\hat F_n}^+(x))  dx =\intinfinity \int_x^\infty d \hat{F_n}(t) dx.\label{eq:2}
\end{equation}
Since ${\hat F_n}^+(x)$ has bounded support on $\mathbb{R}$ $\forall n$, the integral in \eqref{eq:2} is finite.
Applying Fubini's theorem to the RHS of \eqref{eq:2}, we obtain
\begin{equation}
\intinfinity \int_x^\infty d \hat{F_n}(t) dx = \intinfinity \int_0^t dx d \hat{F_n}(t) = \intinfinity t d\hat{F_n}(t) = \frac{1}{n} \sum_{i=1}^n U^+_{[i]},
 \end{equation}
 where $U^+_{[i]}, i=1,\ldots,n$ denote the order statistics, i.e., $U^+_{[1]} \le \ldots \le U^+_{[n]}$.
 
Now, notice that 

$$
\frac{1}{n}
\sum_{i=1}^n U^+_{[i]}
=
\frac{1}{n}
\sum_{i=1}^n U^+_i
\overset{a.s}\longrightarrow 
E(U^+),
$$
From the foregoing,
$$
\lim_{n\rightarrow \infty} \intinfinity L\cdot(1-\hat{F_n}(x)) dx
\overset{a.s} \longrightarrow
\intinfinity L \cdot(P(U^+>x)) dx.$$
Hence, we have 
$$
\int_0^\infty w^{(+)}(1-\widehat{F_n}(x)) dx \xrightarrow{a.s.} 
\int_0^\infty w^{(+)}(P(U^+)>x) dx.
$$
The claim in \eqref{eq:3} now follows by invoking the generalized dominated convergence theorem by setting $f_n = w^+(1-{\hat F_n}^+(x))$ and $g_n = L\cdot(1-\hat{F_n}(x))$, and noticing that $L\cdot(1-\hat{F_n}(x)) \xrightarrow{a.s.} L(P(U^+>x))$ uniformly $\forall x$. The latter fact is implied by the Glivenko-Cantelli theorem (cf. Chapter 2 of \cite{wasserman2006}).

Following similar arguments, it is easy to show that 
$$
w^-(1-{\hat F_n}^-(x))  dx \rightarrow \intinfinity w^-(P(U^-)>x) dx.
$$
The final claim regarding convergence of $\hat V_n(X)$ to $V(X)$ now follows.
\end{proof}
%%%%%%%%%%%%%%%%%%%%%%%%%%%%%%%%%%%%%%%%%%%%%%%%%%%%%%%%%%%%%%%%%%%%%%%%%%%%%%%%
%%%%%%%%%%%%%%%%%%%%%%%%%%%%%%%%%%%%%%%%%%%%%%%%%%%%%%%%%%%%%%%%%%%%%%%%%%%%%%%%
\subsection*{Proof of Proposition \ref{thm:asymp-conv}: Asymptotic convergence rate}
In order to prove the convergence rate of the policy optimization algorithms, we need a uniform bound on the distance between $V_n(X)$ and the CPT-value $V(X)$, i.e., $||\widehat{V_n}(X)-V(X)||_{\infty}$. For this purpose, Proposition \ref{thm:asymp-conv} had a law of the iterated logarithm type result, which states that $||\widehat{V_n}(X)-V(X)||_{\infty}$ is of the order $O(n^{-1/2})$ (ignoring log-factors) for sufficiently large $n$. 
Before proving this result, we recall the law of the iterated logarithm when empirical distribution function is in one dimension (see \cite{van2000asymptotic} for a detailed description):
\begin{theorem}(\textbf{\textit{Law of the iterated logarithm.}})
\\
Let 
$\hat{F_n}$ denote the empirical distribution and $F$ the true distribution. Then, for sufficiently large $n$, we have 
$$
\limsup_{n\rightarrow \infty} \sqrt{\frac{n}{2log log n}} ||\hat{F_n}-F||_{\infty} \leq \frac{1}{2}, \text{a.s.}
$$
\end{theorem}

\begin{proof}(\textbf{\textit{Proof of Proposition \ref{thm:asymp-conv}: Asymptotic convergence rate}})

Let's focus on the difference 
$
\left|
\intinfinity w^{+}(P(u^+((X)))>x) dx- \intinfinity \wfn dx
\right|
$. We have
\begin{align*}
&
\left|
\intinfinity w^{+}(P(U^+)>x) dx- \intinfinity \wfn dx
\right|
\\
&
=
\left|
\int_0^M w^{+}(P(U^+)>x) dx- \int_0^M \wfn dx
\right|
\\
&
\leq
\left|
\int_0^M L\cdot |P(U^+<x)-{\hat F_n}^+(x)| dx
\right|
\\
&
\leq
LM
\sup_{x\in \mathbb{R}}
\left|
P(U^+<x)-{\hat F_n}^+(x)
\right|.
\end{align*}
Using the law of iterated logarithm, we obtain 
$$
\limsup_{n\rightarrow \infty} \sqrt{\frac{n}{2log log n}} \left\|\intinfinity \wfn- \intinfinity P(U^+>x)dx\right\|_{\infty} 
\leq \frac{1}{2} LM, \text{a.s.}
$$
Along similar lines, we obtain 
$$
\limsup_{n\rightarrow \infty} \sqrt{\frac{n}{2log log n}} \left\|\intinfinity \wfnn- \intinfinity P(U^->x)dx\right\|_{\infty} 
\leq \frac{1}{2} LM, \text{a.s.}
$$
The main claim follows by summing the above two inequalities. 
\end{proof}

%%%%%%%%%%%%%%%%%%%%%%%%%%%%%%%%%%%%%%%%%%%%%%%%%%%%%%%%%%%%%%%%%%%%%%
%%%%%%%%%%%%%%%%%%%%%%%%%%%%%%%%%%%%%%%%%%%%%%%%%%%%%%%%%%%%%%%%%%%%%%
%%%%%%%%%%%%%%%%%%%%%%%%%%%%%%%%%%%%%%%%%%%%%%%%%%%%%%%%%%%%%%%%%%%%%%
%\subsection*{Proof of Proposition \ref{thm:dkw}}
%For proving Proposition \ref{thm:dkw}, we require the following well-known inequality that provide a finite-time bound on the distance between empirical distribution and the true distribution:
%\begin{lemma}{\textbf{\textit{(Dvoretzky-Kiefer-Wolfowitz (DKW) inequality)}}}\\
%Let ${\hat F_n}(x)=\frac{1}{n} \sum_{i=1}^n 1_{((X_i) \leq x)}$ denote the empirical distribution of a r.v. $X$, with $X_1,\ldots,X_n$ being sampled from the true distribution $F(X)$.
%The, for any $n$ and $\epsilon>0$, we have
%$$
%P(\sup_{x\in \mathbb{R}}|\hat{F_n}(x)-F(x)|>\epsilon ) \leq 2 e^{-2n\epsilon^2}.
%$$
%\end{lemma}

For proving Proposition \ref{thm:dkw}, we require the following well-known inequality that provide a finite-time bound on the distance between empirical distribution and the true distribution:
\begin{lemma}{\textbf{\textit{(Dvoretzky-Kiefer-Wolfowitz (DKW) inequality)}}}\\
Let ${\hat F_n}(x)=\frac{1}{n} \sum_{i=1}^n 1_{((X_i) \leq x)}$ denote the empirical distribution of a r.v. $X$, with $X_1,\ldots,X_n$ being sampled from the true distribution $F(X)$.
The, for any $n$ and $\epsilon>0$, we have
$$
P(\sup_{x\in \mathbb{R}}|\hat{F_n}(x)-F(x)|>\epsilon ) \leq 2 e^{-2n\epsilon^2}.
$$
\end{lemma}

The reader is referred to Chapter 2 of \cite{wasserman2006} for more on empirical distributions in general and DKW inequality in particular.

\begin{proof}(\textbf{\textit{Theorem \ref{thm:dkw}}})

Since $U^+$ is bounded above by $M$ and $w^+$ is Lipschitz with constant $L$, we have
\begin{align*}
&\left|\intinfinity w^+(P(U^+)>x) dx- \intinfinity w^+(1- {\hat F_n}^+(x)) dx\right|
\\
= & \left|\int_0^M w^+(P(U^+)>x) dx- \int_0^M w^+(1- {\hat F_n}^+(x)) dx\right|
\\
\leq&
\left|\int_0^M L\cdot |P(U^+<x)-{\hat F_n}^+(x)| dx\right|\\
\leq&
LM\sup_{x\in \mathbb{R}}\left|P(U^+<x)-{\hat F_n}^+(x)\right|.
\end{align*}
Now, plugging in the DKW inequality, we obtain
\begin{align}
&
P\left(\left|\intinfinity w^+(P(U^+)>x) dx- \intinfinity w^+(1- {\hat F_n}^+(x)) dx\right|>\epsilon/2\right)
\nonumber\\
&
\leq
 P\left(LM\sup_{x\in \mathbb{R}} \left|(P(U^+<x)-{\hat F_n}^+(x)\right|>\epsilon/2\right) \leq 2 e^{-n \frac{\epsilon^2}{2 L^2 M^2}}.\label{eq:dkw1}
\end{align}

Along similar lines, we obtain
\begin{align}
&
P\left(\left|\intinfinity w^-(P(U^-)>x) dx- \intinfinity w^-(1- {\hat F_n}^-(x)) dx\right|>\epsilon/2\right)
\leq 2 e^{-n \frac{\epsilon^2}{2 L^2 M^2}}.\label{eq:dkw2}
\end{align}

Combining \eqref{eq:dkw1} and \eqref{eq:dkw2}, we obtain
\begin{align*}
P(|\widehat V_n(X) - V(X)|>\epsilon) 
&\le P\left(\left|\intinfinity w^+(P(U^+)>x) dx- \intinfinity w^+(1- {\hat F_n}^+(x)) dx\right|>\epsilon/2\right) \\
&+ 
P\left(\left|\intinfinity w^-(P(U^-)>x) dx- \intinfinity w^-(1- {\hat F_n}^-(x)) dx\right|>\epsilon/2\right)\\
&\le 4 e^{-n \frac{\epsilon^2}{2 L^2 M^2}}.
\end{align*} 
And the claim follows. 
\end{proof}


%%%%%%%%%%%%%%%%%%%%%%%%%%%%%%%%%%%%%%%%%%%%%%%%%%%%%%%%%%%%%%%%%%%%%%%%%%%%%%%%
%%%%%%%%%%%%%%%%%%%%%%%%%%%%%%%%%%%%%%%%%%%%%%%%%%%%%%%%%%%%%%%%%%%%%%%%%%%%%%%%
\subsubsection{\holder continuous weights}
\label{sec:holder-proofs}
\begin{proposition}
\label{prop:Holder-cpt-finite}
Under (A1'), the CPT-value $V(X)$ as defined by \eqref{eq:cpt-mdp} is finite. 
\end{proposition}
\begin{proof}

H\"{o}lder continuity of $w^+$ implies that 
$$
\int_0^{+\infty} w^+(P(U^+>t)) dz 
< \int_0^{+\infty} P^{\alpha} (U^+>z) dz
< \int_0^{+\infty} P^{\gamma} (U^+>z) dz 
<+\infty.
$$
The second inequality is valid since $P(U^+>z) \leq 1$. The claim follows for the first integral in \eqref{eq:cpt-mdp} and the finiteness of the second integral in \eqref{eq:cpt-mdp} can be argued in an analogous fashion.
\end{proof}

%%%%
\begin{proposition}
\label{prop:holder-quantile}
Assume (A1'). Then, we have 
\begin{align}
\label{eq:simple-estimation}
\begin{split}
\lim_{n \rightarrow \infty} \sum_0^{n-1} \xi^+_{\frac{i}{n}} (w^+(\frac{n-i}{n})- w^+(\frac{n-i-1}{n}) ) = \int_0^{+\infty} w^+(P(U^+>z)) dz < +\infty,
\\
\lim_{n \rightarrow \infty} \sum_0^{n-1} \xi^-_{\frac{i}{n}} (w^-(\frac{n-i}{n})- w^-(\frac{n-i-1}{n}) ) = \int_0^{+\infty} w^-(P(U^->z)) dz < +\infty
\end{split}
\end{align}
and also

with the variable $\xi^+_{\frac{i}{n}}$ and $\xi^-_{\frac{i}{n}}$ denote respectively the $\frac{i}{n}$th quantile of the 
r.v. $U^+$ and $U^-$.
\end{proposition}

\todoj[inline]{Substitute $U$, $t dt$, $w$, $\xi$ with $U^+$, $z dz$, $w^+$ and $\xi^+$, and say similar argument holds for $-$ parts,done}
\begin{proof}
We shall focus on proving the first part of equation (28). Consider the following linear combination of simple functions: 
\begin{align}
\sum_{i=0}^{n-1} w^+_{\frac{i}{n}} 
\cdot I_{[\xi^+_\frac{n-i-1}{n}, \xi_\frac{n-i}{n}]}(t),
\end{align}
which will converge almost everywhere to the function $w(P(U>t))$ in the interval $[0, +\infty)$, and also notice that 
\begin{align}
\sum_{i=0}^{n-1} w^+_{\frac{i}{n}} 
\cdot I_{[\xi^+_\frac{n-i-1}{n}, \xi^+_\frac{n-i}{n}]}(t)
<
w(P(U>t)),
\text{         } \forall t \in [0,+\infty)
\end{align}

The integral of (4) equals to 

\begin{align}
& \int_0^{+\infty} \sum_{i=0}^{n-1} w^+_{\frac{i}{n}} \cdot I_{[\xi^+_\frac{n-i-1}{n},
\xi^+_\frac{n-i}{n}]}(t) \\ & = \sum_{i=0}^{n-1} w^+_{\frac{i}{n}}(t) \cdot (\xi^+_{\frac{n-i}{n}} -
\xi^+_{\frac{n-i-1}{n}}) \\ & = \sum_{i=0}^{n-1} \xi^+_{\frac{i}{n}} \cdot (w^+_{\frac{n-i}{n}}-
    w^+_{\frac{n-i-1}{n}})
\end{align}

The H\"{o}lder continuity property assures the fact that 
$\lim_{n \rightarrow \infty}  | w^+(\frac{n-i}{n})- w^+(\frac{n-i-1}{n})| =0$, and the limit in (28) holds through a typical application of dominated convergence theorem.
The second part of equation (28) can be justified in a similar fashion.
\end{proof} 
%%%

%%%%%%%%%\xi^+
\subsection*{Proof of Proposition \ref{prop:holder-asymptotic}}
\todoj[inline]{Substitute $U$, $t dt$, $w$, $\xi^+$ with $U^+$, $z dz$, $w^+$ and $\xi^+$, resply and say similar argument holds for $-$ parts, done}
\begin{proof}
We would prove the $w^+$ part, and the $w^-$ part would be proved in a similar fashion.

The main part of the proof is concentrated on finding an upper bound of the probability
\begin{align}
P ( \left| \sum_{i=1}^{n-1} U^+_{[i]} \cdot (w^+(\frac{n-i}{n} )  - w^+(\frac{n-i-1}{n} ) ) -
\sum_{i=1}^{n-1} \xi^+_{\frac{i}{n}} \cdot (w^+(\frac{n-i}{n} )  - w^+(\frac{n-i-1}{n} ) ) \right| >
\epsilon),
\end{align}
with $\epsilon$
being any given constant $\epsilon>0$.
Observe the fact that
\begin{align*}
& P ( \left| \sum_{i=1}^{n-1} U^+_{[i]} \cdot (w^+(\frac{n-i}{n} )  - w^+(\frac{n-i-1}{n} ) ) -
\sum_{i=1}^{n-1} \xi^+_{\frac{i}{n}} \cdot (w^+(\frac{n-i}{n} )  - w^+(\frac{n-i-1}{n} ) ) \right| >
\epsilon) \\ & \leq P ( \bigcup _{i=1}^{n-1} \{ \left| U^+_{[i]} \cdot (w^+_{(\frac{n-i}{n})} -
w^+{(\frac{n-i-1}{n})}) - \xi^+_{\frac{i}{n}} \cdot (w^+(\frac{n-i}{n} )  - w^+(\frac{n-i-1}{n} ) )
\right| > \frac{\epsilon}{n} \}) \\ & \leq \sum _{i=1}^{n-1} P ( \left| U^+_{[i]} \cdot
(w^+(\frac{n-i}{n} )  - w^+(\frac{n-i-1}{n} ) ) - \xi^+_{\frac{i}{n}} \cdot (w^+_{(\frac{n-i}{n})} -
w^+_{(\frac{n-i-1}{n})}) \right| > \frac{\epsilon}{n}) \\ & = \sum _{i=1}^{n-1} P ( \left| ( U^+_{[i]} -
\xi^+_{\frac{i}{n}}) \cdot (w^+(\frac{n-i}{n} )  - w^+(\frac{n-i-1}{n} ) ) \right| > \frac{\epsilon}{n})
\\ & \leq \sum _{i=1}^{n-1} P ( \left| ( U^+_{[i]} - \xi^+_{\frac{i}{n}}) \cdot (\frac{1}{n})^{\alpha}
\right| > \frac{\epsilon}{n}) \\ & = \sum _{i=1}^{n-1} P ( \left| ( U^+_{[i]} - \xi^+_{\frac{i}{n}})
\right| > \frac{\epsilon}{n^{1-\alpha}})
\end{align*}
Now we are turning to find the upper bound of the probability of a single item in the sum above, i.e,
\begin{align*}
& P( \left | U^+_{[i]} - \xi^+_{\frac{i}{n}} \right | > \frac {\epsilon} {n^{(1-\alpha)}}) \\ & = P (
    U^+_{[i]} - \xi^+_{\frac{i}{n}} > \frac {\epsilon} {n^{(1-\alpha)}}) + P ( U^+_{[i]} -
    \xi^+_{\frac{i}{n}} < - \frac {\epsilon} {n^{(1-\alpha)}}).
\end{align*} 

\noindent We focus on the term 
$
P ( U^+_{[i]} - \xi^+_{\frac{i}{n}} > \frac {\epsilon}{\nalpha})
$.
We denote $W_t= I_{(U^+_t > \xi^+_{\frac{i}{n}} + \frac{\epsilon}{n^{(1-\alpha)}})}$
,t as a dummy value. According to the non-decreasing property of the probability distribution function, we know that 
\begin{align*}
& P ( U^+_{[i]} - \xi^+_{\frac{i}{n}} > \frac {\epsilon}{\nalpha}) \\ & = P ( \sum _{t=1}^{n} W_t >
n\cdot(1-\frac{i}{n^{(1-\alpha)}})) \\ & = P ( \sum _{t=1}^{n} W_t - n \cdot [1-F(\xi^+_{\frac{i}{n}}
+\frac{\epsilon}{n^{(1-\alpha)}})] > n \cdot [F(\xi^+_{\frac{i}{n}} +\frac{\epsilon}{n^{(1-\alpha)}})
- \frac{i}{n}]).
\end{align*}

\noindent Notice that 
$E W_t = 1-F(\xi^+_{\frac{i}{n}} +\frac{\epsilon}{n^{(1-\alpha)}})$, and by recalling Hoeffding's inequality, one can derive that

\begin{align}
P ( \sum _{i=1}^{n} W_t - n \cdot [1-F(\xi^+_{\frac{i}{n}} +\frac{\epsilon}{n^{(1-\alpha) }  } ) ] > n
\cdot [F(\xi^+_{\frac{i}{n}} +\frac{\epsilon}{n^{(1-\alpha)} } ) - \frac{i}{n}]) < e^{-2n\cdot
\delta^{'}_t},
\end{align}
with $\delta^{'}_i = F(\xi^+_{\frac{i}{n}} +\frac{\epsilon} {n^{(1-\alpha)} }) - \frac{i}{n}$, and if 
$F(x)$ is Lipschitz, $ \delta^{'}_i \leq L \cdot (\frac{\epsilon}{\nalpha})$.
Therefore we will have
\begin{align}
P ( U^+_{[i]} - \xi^+_{\frac{i}{n}} > \frac {\epsilon}{\nalpha}) < e^{-2n\cdot L
\frac{\epsilon}{\nalpha} } = e^{-2n^\alpha \cdot L }
\end{align}
Through the similar fashion, one can show that 
\begin{align*}
P ( U^+_{[i]} -\xi^+_{\frac{i}{n}} < -\frac {\epsilon} {\nalpha}) \leq e^{-2n^\alpha \cdot L }
\end{align*}
Subject to the assumption that $F(x)$ is Lipschitz continuous, one can state that 
\begin{align*}
P ( \left| U^+_{[i]} -\xi^+_{\frac{i}{n}} \right| < -\frac {\epsilon} {\nalpha}) \leq 2\cdot
e^{-2n^\alpha \cdot L } , \text{   }\forall i\in \mathbb{N} \cap (0,1) 
\end{align*}
As a result we can derive a bound for the probability in (6) such that
\begin{align}
&
P ( \left| \sum_{i=1}^{n-1} U^+_{[i]} \cdot (w^+(\frac{n-i}{n} )  - w^+(\frac{n-i-1}{n} ) ) -
\sum_{i=1}^{n-1} \xi^+_{\frac{i}{n}} \cdot (w^+(\frac{n-i}{n} )  - w^+(\frac{n-i-1}{n} ) ) \right| >
\epsilon) \leq 2n\cdot e^{-2n^\alpha \cdot L}.\label{eq:holder-sample-complexity-extract}
\end{align}

Notice that $\sum_{n=1}^{+\infty}  2n \cdot e^{-2n^{\alpha}\cdot L}< \infty$ since the sequence 
$2n \cdot e^{-2n^{\alpha}\cdot L}$ will decrease rapidly than the sequence
$\frac{1}{n^k}$, $\forall k>1$.

By applying Borel Cantelli lemma,
$$
P ( \left| \sum_{i=1}^{n-1} U^+_{[i]} \cdot (w^+(\frac{n-i}{n} )  - w^+(\frac{n-i-1}{n} ) ) -
\sum_{i=1}^{n-1} \xi^+_{\frac{i}{n}} \cdot (w^+(\frac{n-i}{n} )  - w^+(\frac{n-i-1}{n} ) ) \right| >
\epsilon , i.o.) =0 , \text{   } \forall \epsilon >0 $$
which implies 
$$
\sum_{i=1}^{n-1} U^+_{[i]} \cdot (w^+(\frac{n-i}{n} )  - w^+(\frac{n-i-1}{n} ) ) - \sum_{i=1}^{n-1}
\xi^+_{\frac{i}{n}} \cdot (w^+(\frac{n-i}{n} )  - w^+(\frac{n-i-1}{n} ) ) \xrightarrow{n \rightarrow
+\infty} 0 \text{   w.p } 1 ,
$$
which constitute the proof of theorem's statement, that 
\begin{align}
\lim_{n\rightarrow +\infty} \sum_{i=1}^{n-1} U^+_{ [i ] } (w^+(\frac{n-i+1}{n})- w^+(\frac{n-i}{n}))
&\xrightarrow{n \rightarrow\infty} \int_0^{+\infty} w^+(P(U>t)) dt, \text{w.p} 1
\end{align}
\end{proof}

%%%%%%%%%%%%%% Sample complexity result
<<<<<<< HEAD





\begin{proof}
Follows from \eqref{eq:holder-sample-complexity-extract} in the proof of Proposition \ref{prop:holder-asymptotic}.
\end{proof}
=======
% \subsection*{Proof of Proposition \ref{prop:holder-sample-complexity}}
% \begin{proof}
% Follows from \eqref{eq:holder-sample-complexity-extract} in the proof of Proposition \ref{prop:holder-asymptotic}.
% \end{proof}
>>>>>>> origin/master



\subsection*{Proof of Proposition \ref{prop:holder-dkw}}
\begin{proof}
We would prove the $w^+$ part, and the $w^-$ part will be proved through a similar fashion.
Since $U^+$ is bounded above by $M$ and $w$ is H\"{o}lder with constant $C$ and power $\alpha$, we have
\begin{align*}
&\left|\int_0^{\infty} w^+(P(U^+)>t) dt- \int_0^{\infty} w^+(1- {\hat F^+_n}(t)) dt\right| \\ = &
    \left|\int_0^M w^+(P(U^+)>t) dt- \int_0^M w^+(1- {\hat F^+_n}(t)) dt\right| \\
\leq& \left|\int_0^M C\cdot |P(U^+<t)-{\hat F^+_n}(t)|^\alpha dt\right|\\ \leq& LC\sup_{x\in
\mathbb{R}}\left|P(U^+<t)-{\hat F^+_n}(t)\right|^\alpha.
\end{align*}
Now, plugging in the DKW inequality, we obtain
\begin{align}
&
P\left(\left|\intinfinity w^+(P(U^+)>t) dt- \intinfinity w^+(1- {\hat F^+_n}(t)) dt\right|>\epsilon\right)
\nonumber\\
& \leq P\left(LM\sup_{t\in \mathbb{R}} \left|(P(U^+<t)-{\hat F^+_n}(t)\right|^\alpha>\epsilon\right)
\leq  e^{-n \frac{\epsilon ^{(2/\alpha)}} {2 L^2 M^2}}.\label{eq:dkw3}
\end{align}
\end{proof}

\subsubsection{Proofs for discrete value $X$}
\label{sec:proofs-discrete}
For the sake of notational convenience, we assume $w^+=w^-=w$. 
For proving Proposition \ref{prop:sample-complexity-discrete}, we require Hoeffding's inequality, which is given below.
\begin{lemma}
Let $Y_1,...Y_n$ be \emph{independent random variables} satisfying $P(a\leq Y_i \leq b)= 1,$ each i, where $a<b.
$Then for $t>0$,
$$P(\left|\sum_{i=1}^n Y_i -\sum_{i=1}^n E(Y_i)\right| \geq nt ) \leq 2\exp{\{-2nt^2 /(b-a)^2\}} $$
\end{lemma}


Let
\begin{align}
\label{eq:hatFk}
\hat F_k = 
\begin{cases}
   \sum_{i=1}^k \hat p_k & \text{if   } k \leq l \\
   \sum_{i=k}^K \hat p_k & \text{if  }  k > l.
\end{cases}  
\end{align}

The following proposition gives the rate at which $\hat{F_k}$ converges to $F_k$.
\begin{proposition}
\label{prop:hoeffding-discrete}
Let $F_k$ and $\hat F_k$ be as defined in \eqref{eq:Fk}--\eqref{eq:hatFk}, Then, we have that, for every $\epsilon >0$, 
$$P(|\hat{F_k}-F_k| > \epsilon) \leq 2 e^{-2n \epsilon^2} $$
\end{proposition}
\begin{proof}
We focus on the case when $k > l$,and the case of $k \leq l$ will be proved through the same fashion.
Notice that when $k>l$,  $\hat F_k =I_{(U_i \geq  x_k) }$ and the random variables are independent to each other for each i, and it is bounded by 1. 
The probability $P(\left|F_k- F_k \right| > \epsilon)$ is equal to 
\begin{align*}
&
P(\left|\hat{F_k}- F_k \right| > \epsilon) \\ & = P(\left| \frac{1}{n} \sum_{i=1}^n I_{\{U_i \geq
x_k\}} - \frac{1}{n} \sum_{i=1}^n E(I_{\{U_i \geq x_k\}}) \right| > \epsilon) \\ & = P(\left|
\sum_{i=1}^n I_{\{U_i \geq x_k\}} - \sum_{i=1}^n E(I_{\{U_i \geq x_k\}}) \right| > n\epsilon) \\ &
    \leq 2e^{-2n \epsilon^2}
\end{align*}
\end{proof}

The proof of Proposition \ref{prop:sample-complexity-discrete} requires the following claim which gives the convergence rate under local Lipschitz weights. 
\begin{proposition}
\label{prop:discrete-first-term}
Under conditions of Proposition \ref{prop:sample-complexity-discrete}, with $F_k$ and $\hat F_k$ as defined in \eqref{eq:Fk}--\eqref{eq:hatFk},  we have
$$P(\left| \sum_{i=1}^K x_k w(\hat{F_k}) - \sum_{i=1}^K x_k w(F_k) \right| >\epsilon) < K\cdot (
e^{-\delta^2\cdot 2n} + e^{-\epsilon^2 2n/(KLA)^2}) $$ 
\end{proposition}

\begin{proof}
Observe that

\begin{align*}
&
P(\left| \sum_{k=1}^K x_k w(\hat{F_k}) - \sum_{k=1}^K x_k w(F_k) \right| >\epsilon) \\ & = P (
\bigcup_{k=1}^K \left| x_k w(\hat{F_k}) -x_k w(F_k) \right| > \frac {\epsilon} {K}) \\ & \leq
    \sum_{k=1}^K P (\left| x_k w(\hat{F_k}) -x_k w(F_k) \right| > \frac {\epsilon} {K})
\end{align*}
Notice that $\forall k =1,....K$
$[{p_k}- \delta, {p_k}+\delta)$,
the function $w$ is locally Lipschitz with common constant $L$.
Therefore, for each k, we can decompose the probability as 
\begin{align*}
& P (\left| x_k w(\hat{F_k}) -x_k w(F_k) \right| > \frac {\epsilon} {K}) \\ & = P ( [ \left| F_k -
\hat{F_k} \right| >\delta ] \bigcap [ \left| x_k w(\hat{F_k}) -x_k w(F_k) \right| ] > \frac
{\epsilon} {K}) + P ( [ \left| F_k - \hat{F_k} \right| \leq\delta ] \bigcap [ \left| x_k
    w(\hat{F_k}) -x_k w(F_k) \right| ] > \frac {\epsilon} {K}) \\ & \leq P ( \left| F_k - \hat{F_k}
    \right| >\delta) + P ( [ \left| F_k - \hat{F_k} \right| \leq\delta ] \bigcap [ \left| x_k
    w(\hat{F_k}) -x_k w(F_k) \right| ] > \frac {\epsilon} {K})
\end{align*}
 
According to the property of locally Lipschitz continuous,
we have
\begin{align*}
& P ( [ \left| F_k - \hat{F_k} \right| \leq\delta ] \bigcap [ \left| x_k w(\hat{F_k}) -x_k w(F_k)
\right| ] > \frac {\epsilon} {K}) \\ & \leq P(x_k L \left| F_k - \hat{F_k} \right| > \frac
    {\epsilon} {K}) \leq e^ {-\epsilon\cdot 2n /(K L x_k)^2} \leq e^ {-\epsilon\cdot 2n /(K L A)^2}
    \text{     for    } \forall k
\end{align*}
And similarly,
\begin{align*}
& P(\left| F_k - \hat{F_k} \right| > \delta) \\ & \leq e^{-\delta^2 /2n} \text{    for     } \forall
    n
\end{align*}
And as a result,
\begin{align*}
& P(\left| \sum_{k=1}^K x_k w(\hat{F_k}) - \sum_{k=1}^K x_k w(F_k) \right| >\epsilon) \\ & \leq
\sum_{k=1}^K P (\left| x_k w(\hat{F_k}) -x_k w(F_k) \right| > \frac {\epsilon} {K}) \\ & \leq
             \sum_{k=1}^K e^{-\delta^2\cdot 2n} + e^{-\epsilon^2 \cdot 2n/ (KLA)^2} \\ & =K\cdot
    (e^{-\delta^2\cdot 2n} + e^{-\epsilon^2 \cdot 2n/ (KLA)^2})
\end{align*}
\end{proof}

\subsection*{Proof of Proposition \ref{prop:sample-complexity-discrete}}
\begin{proof}
We need to prove that
\begin{align}
P(\left|\sum_{i=1}^K u(x_k) \cdot(w(\hat{F_k})- w(\hat F_{k+1}) )
-  
\sum_{i=1}^K u(x_k) \cdot(w(F_k)- w(F_{k+1}) )
\right| \leq \epsilon) > 1-\rho
\text{      ,     } \forall n> \frac{\ln(\frac{4K}{a})} { M}, 
\end{align}
where $w$ is Locally Lipschitz continuous with constants $L_1,....L_K$ on the points $F_1,....F_K$.
From a parallel argument to that in the proof of Proposition \ref{prop:discrete-first-term}, it is easy to infer that
$$P(\left| \sum_{i=1}^K x_k w(\hat F_{k+1}) - \sum_{i=1}^K x_k w(F_{k+1}) \right| >\epsilon) <
K\cdot ( e^{-\delta^2\cdot 2n} + e^{-\epsilon^2 2n/(KLA)^2})
$$
Hence,
\begin{align*}
& P(\left|\sum_{i=1}^K x_k \cdot(w(\hat{F_k})- w(\hat F_{k+1}) ) -  \sum_{i=1}^K x_k \cdot(w(F_k)-
w(F_{k+1}) ) \right| > \epsilon) \\ & \leq P(\left|\sum_{i=1}^K x_k \cdot(w(\hat{F_k})) -
    \sum_{i=1}^K x_k \cdot(w(F_k)) \right| > \epsilon/2) + P(\left|\sum_{i=1}^K x_k
    \cdot(w(\hat F_{k+1})) -  \sum_{i=1}^K x_k \cdot(w(F_{k+1})) \right| > \epsilon/2) \\ & \leq 2K
    (e^{-\delta^2\cdot 2n} + e^{-\epsilon^2 2n/(KLA)^2})
\end{align*}
The main claim follows.
\end{proof}


%%%%%%%%%%%%%%%%%%%%%%%%%%%%%%%%%%%%%%%%%%%%%%%%%%%%%%%%%%%%%%%%%%%%%%%%%%%%%%%%
%%%%%%%%%%%%%%%%%%%%%%%%%%%%%%%%%%%%%%%%%%%%%%%%%%%%%%%%%%%%%%%%%%%%%%%%%%%%%%%%
\subsection{Proofs for PG-CPT-SPSA}
\label{appendix:1spsa}

To prove the main result in Theorem \ref{thm:1spsa-conv}, we first show, in the following lemma, that the gradient estimate using SPSA is only an order $O(\delta_n^2)$ term away from the true gradient. The proof differs from the corresponding claim for regular SPSA (see Lemma 1 in \cite{spall}) since we have a non-zero bias in the function evaluations, while the regular SPSA assumes noise is zero-mean. Following this lemma, we complete the proof of Theorem \ref{thm:1spsa-conv} by invoking the well-known Kushner-Clark lemma \cite{kushner-clark}.

\begin{lemma}
\label{lemma:1spsa-bias}
Let $\F_n = \sigma(\pi_m,m\le n)$, $n\ge 1$.
Then, for any $i = 1,\ldots,\L\M$, we have almost surely,  
\begin{align}
\left| \E\left[\left.\dfrac{\widehat V_n^{\pi_n + \delta_n \Delta_n}(x^0) -\widehat V_n^{\pi_n - \delta_n \Delta_n}(x^0)}{2 \delta_n \Delta_n^{i}}\right| \F_n \right] - \nabla_{i} V^{\pi_n}(x^0) \right| \rightarrow 0 \text{ as } n\rightarrow\infty.
\end{align} 
\end{lemma}
\begin{proof}
Recall that the CPT-value estimation scheme is biased, i.e., providing samples with policy $\pi$, we obtain its CPT-value estimate as $V^{\pi}(x_0) + \epsilon^\pi$. Here $\epsilon^\pi$ denotes the bias. 
% However, since the number of samples $m_n$ go to infinity, we have $$\epsilon^{\pi_n}  \rightarrow 0 \text{ as } n \rightarrow \infty.$$

We claim
\begin{align}
\quad\E\left[\dfrac{\widehat V_n^{\pi_n + \delta_n \Delta_n}(x^0) -\widehat V_n^{\pi_n - \delta_n \Delta_n}(x^0)}{2 \delta_n \Delta_n^{i}} \left.\right| \F_n\right] 
= \quad&\E\left[\dfrac{V_n^{\pi_n + \delta_n \Delta_n}(x^0) -V_n^{\pi_n - \delta_n \Delta_n}(x^0)}{2 \delta_n \Delta_n^{i}} \left.\right| \F_n\right] + \E\left[ \eta_n \mid \F_n\right],
\end{align}
where $\eta_n = \left(\dfrac{\epsilon^{\pi_n +\delta_n\Delta} - \epsilon^{\pi_n-\delta_n\Delta}}{2\delta_n\Delta_n^{i}}\right)$
 is the bias arising out of the empirical distribution based CPT-value estimation scheme.
From Proposition \ref{thm:asymp-conv}, we see that, $\epsilon^\pi = LM\sqrt{\frac{2 \log \log m_n}{m_n}}$, provided $m_n$ is sufficiently large.
Thus, $\eta_n = O\left( \sqrt{\frac{2 \log \log m_n}{m_n}} \frac{1}{\delta_n}\right)$ and since $\frac{1}{\sqrt{m_n} \delta_n} \rightarrow 0$ by assumption (A3) in the main paper, we have that
$\eta_n$ goes to zero asymptotically. In other words,
\begin{align}
\quad\E\left[\dfrac{\widehat V_n^{\pi_n + \delta_n \Delta_n}(x^0) -\widehat V_n^{\pi_n - \delta_n \Delta_n}(x^0)}{2 \delta_n \Delta_n^{i}} \left.\right| \F_n\right] 
&\xrightarrow{n\rightarrow\infty}  \E\left[\dfrac{V^{\pi_n + \delta_n \Delta_n}(x^0) -V^{\pi_n - \delta_n \Delta_n}(x^0)}{2 \delta_n \Delta_n^{i}} \left.\right| \F_n\right].  \label{eq:l1}
\end{align}
We now analyse the RHS of the above.
By using suitable Taylor's expansions,
\begin{align*}
V^{\pi_n + \delta_n \Delta_n}(x^0) = V^{\pi_n}(x^0) + \delta_n \Delta_n\tr \nabla V^{\pi_n}(x^0) + \frac{\delta^2}{2} \Delta_n\tr \nabla^2 V^{\pi_n}(x^0)\Delta_n + O(\delta_n^3), \\
V^{\pi_n - \delta_n \Delta_n}(x^0) = V^{\pi_n}(x^0) - \delta_n \Delta_n\tr \nabla V^{\pi_n}(x^0) + \frac{\delta^2}{2} \Delta_n\tr \nabla^2 V^{\pi_n}(x^0)\Delta_n + O(\delta_n^3).\end{align*}
From the above, it is easy to see that 
\begin{align*}
\dfrac{V^{\theta_n + \delta_n \Delta_n}(x^0) - V^{\theta_n - \delta_n \Delta_n}(x^0)}{2 \delta_n \Delta_n^i}
- \nabla_i V^{\theta_n}(x^0)
=\underbrace{\sum_{j=1,j\not=i}^{N} \frac{\Delta_n^j}{\Delta_n^i}\nabla_j V^{\theta_n}(x^0)}_{(I)} + O(\delta_n^2).
\end{align*}
Taking conditional expectation on both sides, we obtain
\begin{align}
\E\left[\dfrac{V^{\theta_n + \delta_n \Delta_n}(x^0) - V^{\theta_n - \delta_n \Delta_n}(x^0)}{2 \delta_n \Delta_n^i} \left.\right| \F_n\right] 
=& \nabla_i V^{\theta_n}(x^0) + \E\left[\sum_{j=1,j\not=i}^{N} \frac{\Delta_n^j}{\Delta_n^i}\right]\nabla_j V^{\theta_n}(x^0) + O(\delta_n^2)\nonumber\\
= & \nabla_i V^{\theta_n}(x^0) + O(\delta_n^2).\label{eq:l2}
\end{align}
The first equality above follows from the fact that $\Delta_n$ is distributed according to a $\L\M$-dimensional vector of Rademacher random variables and is independent of $\F_n$. The second inequality follows by observing that $\Delta_n^i$ is independent of $\Delta_n^j$, for any $i,j =1,\ldots,\L\M$, $j\ne i$. 

The claim follows by using the fact that $\delta_n \rightarrow 0$ as $n\rightarrow \infty$.
\end{proof}

\subsection*{Proof of Theorem \ref{thm:1spsa-conv}}

\begin{proof}

%Recall that $\F_n = \sigma(\pi_m,m\le n;\Delta_m,m<n)$, $n\ge 1$.
We first rewrite the update rule \eqref{eq:theta-update} as follows: For $i=1,\ldots,\L\M$,
\begin{align}
\pi^{i}_{n+1}  =  \pi^{i}_n -  \gamma_n(\nabla_{i} V^{\pi_n}(x_0) + \beta_n + \xi_n), 
\label{eq:1spsa-equiv}
\end{align}
where 
\begin{align*}
\beta_n \quad= &\quad \E\left(\dfrac{(\widehat V^{\pi_n + \delta_n \Delta_n}(x^0) -\widehat V^{\pi_n - \delta_n \Delta_n}(x^0))}{2 \delta_n \Delta_n^{i}} \mid \F_n \right) - \nabla V^\pi(x^0), \text{ and}\\
\xi_n \quad = & \quad\left(\dfrac{\widehat V^{\pi_n + \delta_n \Delta_n}(x^0) -\widehat V^{\pi_n - \delta_n \Delta_n}(x^0)}{2 \delta_n\Delta_n^{i}}\right)  - \E\left(\dfrac{(\widehat V^{\pi_n + \delta_n \Delta_n}(x^0) -\widehat V^{\pi_n - \delta_n \Delta_n}(x^0))}{2 \delta_n \Delta_n^{i}}\mid \F_n \right).
\end{align*}
In the above, $\beta_n$ is the bias in the gradient estimate due to SPSA and $\xi_n$ is a martingale difference sequence..
% , while  Lemma \ref{lemma:1spsa-bias} and the fact that $\delta_n \rightarrow 0$ imply that $\beta_n$ vanishes asymptotically. 

Convergence of \eqref{eq:1spsa-equiv} can be inferred from Theorem 5.3.1 on pp. 191-196 of \cite{kushner-clark}, provided we verify the necessary assumptions given as (B1)-(B5) below:
\begin{enumerate}[\bfseries (B1)]
\item $\nabla V^\pi(x^0)$ is a continuous $\R^{\L\K}$-valued function.
\item  The sequence $\beta_n,n\geq 0$ is a bounded random sequence with
$\beta_n \rightarrow 0$ almost surely as $n\rightarrow \infty$.
\item The step-sizes $\gamma_n,n\geq 0$ satisfy
$  \gamma_n\rightarrow 0 \mbox{ as }n\rightarrow\infty \text{ and } \sum_n \gamma_n=\infty.$
\item $\{\xi_n, n\ge 0\}$ is a sequence such that for any $\epsilon>0$,
\[ \lim_{n\rightarrow\infty} P\left( \sup_{m\geq n}  \left\|
\sum_{k=n}^{m} \gamma_k \xi_k\right\| \geq \epsilon \right) = 0. \]
\item There exists a compact subset $K$ which is the set of asymptotically stable equilibrium points for the following ODE:
\begin{align}
\dot\pi^{i}_t = \check\Gamma_{i}\left( \nabla V^{\pi^{i}_t}(x^0)\right), \text{ for} i=1,\dots,\L\M,\label{eq:pi-ode}
\end{align}
\end{enumerate} 

In the following, we verify the above assumptions for the recursion \eqref{eq:theta-update}:
\begin{itemize}
\item (B1) holds by assumption in our setting.

\item Lemma \ref{lemma:1spsa-bias} above establishes that the bias $\beta_n$ is $O(\delta_n^2)$ and since $\delta_n \rightarrow 0$ as $n\rightarrow \infty$, it is easy to see that (B2) is satisfied for $\beta_n$. 

\item (B3) holds by assumption (A3) in the main paper.

\item We verify (B4) using arguments similar to those used in \cite{spall} for the classic SPSA algorithm:\\
We first recall Doob's martingale inequality (see (2.1.7) on pp. 27 of \cite{kushner-clark}):
\begin{align}
P\left( \sup_{m\geq 0}   \left\|W_l\right\| \geq \epsilon \right) \le \dfrac{1}{\epsilon^2} \lim_{l\rightarrow \infty} \E \left\|W_l\right\|^2. 
\end{align}
Applying the above inequality to the martingale sequence $\{W_l\}$, where  $W_l := \sum_{n=0}^{l-1} \gamma_n \eta_n$, $l\ge 1$, we obtain
\begin{align}
P\left( \sup_{l\geq k}   \left\|\sum_{n=k}^{l} \gamma_n \xi_n\right\| \geq \epsilon \right) \le \dfrac{1}{\epsilon^2} \E \left\|
\sum_{n=k}^{\infty} \gamma_n \xi_n\right\|^2 = \dfrac{1}{\epsilon^2} \sum_{n=k}^{\infty} \gamma_n^2 \E\left\| \eta_n\right\|^2. \label{eq:b4}
\end{align}
The last equality above follows by observing that, for $m < n$, $\E(\xi_m \xi_n) = \E(\xi_m \E(\xi_n\mid \F_n))=0$.
We now bound $\E\left\| \xi_n\right\|^2$ as follows:
\begin{align}
\E\left\| \xi_n\right\|^2\le &\E\left(\dfrac{\widehat V^{\pi_n + \delta_n \Delta_n}(x^0) -\widehat V^{\pi_n - \delta_n \Delta_n}(x^0)}{2 \delta_n \Delta_n^i}\right)^2 \label{eq:mi}\\
\le &\left(\left(\E\left(\dfrac{\widehat V^{\pi_n + \delta_n \Delta_n}(x^0)}{2 \delta_n \Delta_n^i}\right)^2\right)^{1/2}
+ \left(\E\left(\dfrac{\widehat V^{\pi_n - \delta_n \Delta_n}(x^0)}{2 \delta_n \Delta_n^i}\right)^2\right)^{1/2}\right)^2 \label{eq:minko}\\
\le &\frac1{4\delta_n^2} \left[ \E\left(\frac1{(\Delta_n^i)^{2+2\alpha_1}}\right) \right]^{\frac{1}{1+\alpha1}} \nonumber\\
& \times \left(\left[\E\left[ (\widehat V^{\pi_n + \delta_n \Delta_n}(x^0))\right]^{2+2\alpha_2}\right]^{\frac{1}{1+\alpha_2}} +
\left[\E\left[ (\widehat V^{\pi_n - \delta_n \Delta_n}(x^0))\right]^{2+2\alpha_2}\right]^{\frac{1}{1+\alpha_2}}\right)\label{eq:holder}\\
\le &\frac1{4\delta_n^2} \left(\left[\E\left[ (\widehat V^{\pi_n + \delta_n \Delta_n}(x^0))\right]^{2+2\alpha_2}\right]^{\frac{1}{1+\alpha_2}} +
\left[\E\left[ (\widehat V^{\pi_n - \delta_n \Delta_n}(x^0))\right]^{2+2\alpha_2}\right]^{\frac{1}{1+\alpha_2}}\right) \label{eq:h2}\\
\le & \frac{C}{\delta_n^2}, \text{ for some } C< \infty. \label{eq:h3}
\end{align}
The inequality in \eqref{eq:mi} uses the fact that, for any random variable $X$, $\E\left\|X -  E[X\mid\F_n]\right\|^2 \le \E X^2$. The inequality in \eqref{eq:minko} follows by the fact that $\E (X+Y)^2 \le \left( (\E X^2)^{1/2} + (\E Y^2)^{1/2}\right)^2$.
The inequality in \eqref{eq:holder} uses Holder's inequality, with $\alpha_1, \alpha_2>0$ satisfying $\frac{1}{1+\alpha_1} + \frac{1}{1+\alpha_2}=1$. 
The equality in \eqref{eq:h2} above follows owing to the fact that $\E\left(\frac1{(\Delta_n^i)^{2+2\alpha_1}}\right)  = 1$ as $\Delta_n^i$ is Rademacher. 
The inequality in \eqref{eq:h3} follows by using the fact that, for any $\pi$, the CPT-value estimate $\widehat V^\pi(x^0) = V^\pi(x^0) + \epsilon^\pi$. We assume a finite state-action spaced SSP (which implies that the costs $\max_{s,a} g(s,a) < \infty$) and consider only \textit{proper} policies (which implies that the total cost $D^\pi(x^0)$ is bounded for any policy $\pi$) and finally, by (A1), the weight functions are Lipschitz - these together imply that $V^\pi(x^0)$ is bounded for any policy $\pi$. The bias $\epsilon^\pi$ is bounded by Proposition \ref{thm:asymp-conv} in the main paper.

Thus, $\E\left\| \xi_n\right\|^2 \le \frac{C}{\delta_n^2}$ for some $C<\infty$. Plugging this in \eqref{eq:b4}, we obtain
\begin{align*}
 \lim_{k\rightarrow\infty} P\left( \sup_{l\geq k}   \left\|\sum_{n=k}^{l} \gamma_n \xi_n\right\|\geq \epsilon \right) \le \dfrac{d C}{\epsilon^2} \lim_{k\rightarrow\infty} \sum_{n=k}^{\infty}  \frac{\gamma_n^2}{\delta_n^2} =0.
\end{align*}
The equality above follows from (A3) in the main paper.
\item The set $\K = \{\pi \mid \check\Gamma_{i} \left(\nabla V^{\pi}(x^0)\right)=0, \forall i=1,\ldots,\L\M\}$ serves as the asymptotically stable attractor for the ODE \eqref{eq:pi-ode}.
\end{itemize} 
The claim follows from Kushner-Clark lemma.
\end{proof}
%%%%%%%%%%%%%%%%%%%%%%%%%%%%%%%%%%%%%%%%%%%%%%%%%%%%%%%%%%%%%%%%%%%%%%%%%%%%%%%%%%%%%%%%%%%%%%%%%%%%%%%%%%%%%%%%%%%%%%%%%%%%%%%
%%%%%%%%%%%%%%%%%%%%%%%%%%%%%%%%%%%%%%%%%%%%%%%%%%%%%%%%%%%%%%%%%%%%%%%%%%%%%%%%%%%%%%%%%%%%%%%%%%%%%%%%%%%%%%%%%%%%%%%%%%%%%%%
%%%%%%%%%%%%%%%%%%%%%%%%%%%%%%%%%%%%%%%%%%%%%%%%%%%%%%%%%%%%%%%%%%%%%%%%%%%%%%%%
%%%%%%%%%%%%%%%%%%%%%%%%%%%%%%%%%%%%%%%%%%%%%%%%%%%%%%%%%%%%%%%%%%%%%%%%%%%%%%%%
\subsection{Proofs for PN-CPT-SPSA}
\label{appendix:2spsa}

Before proving Theorem \ref{thm:2spsa}, we bound the bias in the SPSA based estimate of the Hessian in the following lemma.
\begin{lemma}
\label{lemma:2spsa-bias}
For any $i, j= 1,\ldots,\L\M$, we have almost surely,  
\begin{align}
    \left| \E\left[\left.\dfrac{\widehat V_n^{\pi_n+\delta_n(\Delta_n+\widehat\Delta_n)}(x^0) + \widehat V_n^{\pi_n-\delta_n(\Delta_n+\widehat\Delta_n)}(x^0) - 2 \widehat V_n^{\pi_n}(x^0)}{\delta_n^2 \Delta_n^{i}\widehat\Delta_n^j}\right| \F_n \right] - \nabla^2_{i,j} V^{\pi_n}(x^0) \right| \rightarrow 0 \text{ as } n\rightarrow\infty.
\end{align} 
\end{lemma}
\begin{proof}
As in the proof of Lemma \ref{lemma:1spsa-bias}, we can ignore the bias from the CPT-value estimation scheme and conclude that,
\begin{align}
    \quad&\E\left[\dfrac{\widehat V_n^{\theta_n+\delta_n(\Delta_n+\widehat\Delta_n)}(x^0) + \widehat V_n^{\theta_n-\delta_n(\Delta_n+\widehat\Delta_n)}(x^0) - 2\widehat V_n^{\theta_n}(x^0)}{\delta_n^2 \Delta_n^i\widehat\Delta_n^j} \left.\right| \F_n\right] \nonumber\\
     &\xrightarrow{n\rightarrow\infty}  \E\left[\dfrac{V^{\theta_n+\delta_n(\Delta_n+\widehat\Delta_n)}(x^0) + V^{\theta_n-\delta_n(\Delta_n+\widehat\Delta_n)}(x^0) - 2V^{\theta_n}(x^0)}{\delta_n^2 \Delta_n^i\widehat\Delta_n^j} \left.\right| \F_n\right].  \label{eq:l21}
\end{align}
Now, the RHS of the above approximates the true gradient with only an $O(\delta_n^2)$ error and this can be inferred using arguments similar to that used in the proof of Proposition 4.2 of \cite{bhatnagar2015simultaneous}. We provide the proof here for the sake of completeness.
Using Taylor's expansion as in Lemma \ref{lemma:1spsa-bias}, we obtain
\begin{align*}
&\dfrac{V^{\theta_n+\delta_n(\Delta_n+\widehat\Delta_n)}(x^0) + V^{\theta_n-\delta_n(\Delta_n+\widehat\Delta_n)}(x^0) - 2V^{\theta_n}(x^0)}{\delta_n^2 \Delta_n^i\widehat\Delta_n^j}\\
=&  \frac{(\Delta_n+\hat{\Delta_n})\tr \nabla^2 V^{\theta_n}(x^0)(\Delta_n
+\hat{\Delta_n})}{\triangle_i(n)\hat{\triangle}_j(n)}
+ O(\delta_n^2) \\
=& \sum_{l=1}^{\L\M}\sum_{m=1}^{\L\M} \frac{\Delta_n^l\nabla^2_{l,m}V^{\theta_n}(x^0)\Delta_n^m}{\Delta_n^i
\hat{\Delta}_n^j} + 2\sum_{l=1}^{\L\M}\sum_{m=1}^{\L\M} \frac{\Delta_n^l\nabla^2_{l,m}
V^{\theta_n}(x^0)\hat{\Delta}_n^m}{\Delta_n^i
\hat{\Delta}_n^j}+ \sum_{l=1}^{\L\M}\sum_{m=1}^{\L\M} \frac{\hat{\Delta}_n^l\nabla^2_{l,m}V^{\theta_n}(x^0)\hat{\Delta}_n^m}{\Delta_n^i
\hat{\Delta}_n^j} + O(\delta_n^2).
\end{align*}
Taking conditional expectation, we observe that the first and last term above become zero, while the second term becomes $\nabla^2_{ij}
V^{\theta_n}(x^0)$. The claim follows by using the fact that $\delta_n \rightarrow 0$ as $n\rightarrow \infty$.
\end{proof}

\begin{lemma}
\label{lemma:2spsa-grad}
For any $i = 1,\ldots,\L\M$, we have almost surely,  
\begin{align}
\left| \E\left[\left.\dfrac{\widehat V_n^{\theta_n + \delta_n (\Delta_n + \hat \Delta_n)}(x^0) -\widehat V_n^{\theta_n - \delta_n (\Delta_n + \hat \Delta_n)}(x^0)}{2 \delta_n \Delta_n^i}\right| \F_n \right] - \nabla_i V^{\theta_n}(x^0) \right| \rightarrow 0 \text{ as } n\rightarrow\infty.
\end{align} 
\end{lemma}
\begin{proof}
As in the proof of Lemma \ref{lemma:1spsa-bias}, we can ignore the bias from the CPT-value estimation scheme and conclude that,
\begin{align*}
\quad&\E\left[\dfrac{\widehat V_n^{\theta_n + \delta_n (\Delta_n+\widehat\Delta_n)}(x^0) -\widehat V_n^{\theta_n - \delta_n (\Delta_n+\widehat\Delta_n)}(x^0)}{2 \delta_n \Delta_n^i} \left.\right| \F_n\right] 
\xrightarrow{n\rightarrow\infty}  \E\left[\dfrac{V^{\theta_n + \delta_n \Delta_n}(x^0) -V^{\theta_n - \delta_n \Delta_n}(x^0)}{2 \delta_n \Delta_n^i} \left.\right| \F_n\right].  
\end{align*}
The rest of the proof amounts to showing that the RHS of the above approximates the true gradient with an $O(\delta_n^2)$ correcting term and this can be done in a similar manner as the proof of Lemma \ref{lemma:1spsa-bias}.
\end{proof}

%%%%%%%%%%%%%%%%%%%%%%%%%%%%%%%%%%%%%%%%%%%%%%%%%%%%%%%%%%%%%%%%%%%%%%%%%%%%%%%%%%%%%%%%%%%%%%%%%%%%%%%%%%%%%%%%%%%%%%%%%%%%%%%
%%%%%%%%%%%%%%%%%%%%%%%%%%%%%%%%%%%%%%%%%%%%%%%%%%%%%%%%%%%%%%%%%%%%%%%%%%%%%%%%%%%%%%%%%%%%%%%%%%%%%%%%%%%%%%%%%%%%%%%%%%%%%%%
\subsection*{Proof of Theorem \ref{thm:2spsa}}

Before we prove Theorem \ref{thm:2spsa}, we show that the Hessian recursion \eqref{eq:2spsa-H} in the main paper converges to the true Hessian, for any policy $\pi$.

\begin{lemma}
\label{lemma:h-est}
For any $i, j= 1,\ldots,\L\M$, we have almost surely,  
$$\left \| H^{i, j}_n - \nabla^2_{i,j} V^{\pi_n}(x^0) \right \| \rightarrow 0,
\text{ and }\left \| \Upsilon(\overline H_n)^{-1} - \Upsilon(\nabla^2_{i,j} V^{\pi_n}(x^0))^{-1} \right \| \rightarrow 0.
$$
\end{lemma}
\begin{proof}
 Follows in a similar manner as the proof of Lemmas 7.10 and 7.11 of \cite{Bhatnagar13SR}.
\end{proof}

\begin{proof}\textbf{\textit{(Theorem \ref{thm:2spsa})}}
The proof follows in a similar manner as the proof of Theorem 7.1 in \cite{Bhatnagar13SR} and we provide a sketch below for the sake of completeness.

We first rewrite the recursion \eqref{eq:2spsa} in the main paper as follows:
For $i=1,\ldots, \L\M$
\begin{align}
 \pi^{i}_{n+1} =& \Gamma_{i}\left(\pi^{i}_n - \gamma_n \sum_{j=1}^{\L\M} \bar M^{i,j}(\pi_n) \nabla_{j} V^\pi_n(x^0) + \gamma_n \zeta_n + \chi_{n+1} - \chi_n \right), \label{eq:pi-n}
\end{align}
where 
\begin{align*}
\bar M^{i,j}(\pi) = &\Upsilon(\nabla^2 V^{\pi}(x^0))^{-1}\\
 \chi_n =& \sum_{m=0}^{n-1} \gamma_m \sum_{k=1}^{\L\M} {\bar{M}}_{i,k}(\pi_m)\Bigg(
\frac{V^{\pi_m-\delta_m\Delta_m - \delta_m\widehat\Delta_m}(x^0) -
V^{\pi_m+\delta_m\Delta_m + \delta_m\widehat\Delta_m}(x^0)}{2\delta_m \Delta^k_m} 
 \\
 &- E\left[\frac{V^{\pi_m-\delta_m\Delta_m - \delta_m\widehat\Delta_m}(x^0) -
V^{\pi_m+\delta_m\Delta_m + \delta_m\widehat\Delta_m}(x^0)}{2\delta_m \Delta^k_m} 
\mid {\cal F}_m\right]\Bigg) \text{ and}\\
\zeta_n = &\E\left[\left.\dfrac{\widehat V_n^{\theta_n + \delta_n (\Delta_n + \hat \Delta_n)}(x^0) -\widehat V_n^{\theta_n - \delta_n (\Delta_n + \hat \Delta_n)}(x^0)}{2 \delta_n \Delta_n^i}\right| \F_n \right] - \nabla_i V^{\theta_n}(x^0).
\end{align*}
In lieu of Lemmas \ref{lemma:2spsa-bias}--\ref{lemma:h-est}, it is easy to conclude that $\zeta_n \rightarrow 0$ as $n\rightarrow \infty$, $\chi_n$ is a martingale difference sequence and that $\chi_{n+1} - \chi_n \rightarrow 0$ as $n\rightarrow \infty$. 
Thus, it is easy to see that \eqref{eq:pi-n} is a discretization of the ODE:
$$
\dot\pi^{i}_t = \check\Gamma_{i}\left( \nabla V^{\pi^{i}_t}(x^0) \Upsilon(\nabla^2 V^{\pi_t}(x^0))^{-1} \nabla V^{\pi^{i}_t}(x^0) \right).$$
Further, $\K = \{\pi \mid
\nabla V^{\pi^{i}}(x^0)  \check\Gamma_{i}\left(\Upsilon(\nabla^2 V^{\pi}(x^0))^{-1} \nabla V^{\pi^{i}}(x^0)\right)
=0, \forall i=1,\ldots,\L\M\}$ serves as an asymptotically stable attractor set and the claim follows by applying Kusher-Clark lemma to \eqref{eq:pi-n}.
\end{proof}

\subsection{Proofs for gradient-free policy optimization algorithm}
\label{appendix:mras}

%\textbf{\textit{Natural exponential family (NEF).}} A parameterized family 
%$\left\{f(\cdot,\theta), \theta \in \varTheta \subseteq \Re^m \right\}$
%on $\mathcal{X}$
%is said to to NEF
%if there exist $h:\Re^n\rightarrow \Re$,
%$\Upsilon:\Re^n \rightarrow \Re^m$,
%and $K:\Re^m\rightarrow \Re$ such that
%%
%\begin{equation}\label{eqn:nef}
%f(\mathbf{x},\theta)=\exp\left\{\theta^{T}\Upsilon(\mathbf{x})-K(\theta) \right\}h(\mathbf{x}),~~\forall\, \theta \in \varTheta,
%\end{equation}
%%
%where $K(\theta)=\ln \int_{\mathbf{x}\in \mathcal{X}}\exp\left\{\theta^{T}\Upsilon(\mathbf{x})\right\}h(\mathbf{x})\nu(d\mathbf{x})$, $\varTheta$ is the natural parameter space $\varTheta=\{\theta\in\Re^m:|K(\theta)|<\infty \},$
%and the superscript ``$T$'' denotes the vector transposition.

We begin by remarking that there is one crucial difference between our algorithm and MRAS$_2$ from \cite{chang2013simulation}:
MRAS$_2$ has an expected function value objective, i.e., it aims to minimize a function by using sample observations that have zero-mean noise. On the other hand, the objective in our setting is the CPT-value, which distorts the underlying transition probabilities. The implication here is that MRAS$_2$ can estimate the expected value using sample averages, while we have to resort to integrating the empirical distribution.

Since we obtain samples of the objective (CPT) in a manner that differs from MRAS$_2$, we need to establish that the thresholding step in Algorithm \ref{alg:mras} achieves the same effect as it did in MRAS$_2$. This is achieved by the following lemma, which is a variant of Lemma 4.13 from \cite{chang2013simulation}, adapted to our setting. 

\begin{lemma}\label{lemma:iofo}
    The sequence of random variables $\{\pi^*_n,n=0,1,\ldots \}$ in Algorithm \ref{alg:mras} converges w.p.1 as $n\rightarrow \infty$.
\end{lemma}
\begin{proof}
    Let $\mathcal{A}_n$ be the event that either the first if statement (see \ref{step:3a}) is true or the second if statement in the else clause (see \ref{step:3b}) is true within the Thresholding step of Algorithm \ref{alg:mras}. Let $\mathcal{B}_n:=\{V^{\pi^*_{n}}(x^0)-V^{\pi^*_{n-1}}(x^0)\leq \frac{\varepsilon}{2}\}$.
    Whenever $\mathcal{A}_n$ holds, we have $\widehat V_n^{\pi^*_{n}}(x^0)-\widehat V_n^{\pi^*_{n-1}}(x^0)\geq \varepsilon$ and hence, we obtain

\begin{align*}
P(\mathcal{A}_{n}\cap \mathcal{B}_n) \leq &
P 
\left(\big\{\widehat V_n^{\pi^*_{n}}(x^0)-\widehat V_{n-1}^{\pi^*_{n-1}}(x^0)\geq \varepsilon\big\} \cap \left\{V^{\pi^*_{n}}(x^0)-V^{\pi^*_{n-1}}(x^0)\leq \frac{\varepsilon}{2}\right\}\right)\\
 \leq &
P\Big(\bigcup_{\pi\in\Lambda_n, \pi'\in \Lambda_{n-1}} \Big\{\big\{\widehat V_n^{\pi}(x^0)-\widehat V_{n-1}^{\pi'}(x^0)\geq \varepsilon\big\} \cap \left\{V^\pi(x^0)-V^{\pi'}(x^0)\leq \frac{\varepsilon}{2}\right\} \Big\}\Big)\\
\leq & \sum_{{\pi\in \Lambda_n},{\pi'\in \Lambda_{k-1}}}P\left(\big\{\widehat V_n^{\pi}(x^0)-\widehat V_{n-1}^{\pi'}(x^0)\geq \varepsilon\big\} \cap
\left\{V^\pi(x^0)-V^{\pi'}(x^0)\leq \frac{\varepsilon}{2}\right\}\right) \\
\leq & |\Lambda_n||\Lambda_{n-1}|\sup_{\pi,\pi'\in \Theta}P\left(\big\{\widehat V_n^{\pi}(x^0)-\widehat V_{n-1}^{\pi'}(x^0)\geq \varepsilon\big\} \cap
\left\{V^\pi(x^0)-V^{\pi'}(x^0)\leq \frac{\varepsilon}{2}\right\} \right) \\
\leq &|\Lambda_n||\Lambda_{n-1}|\sup_{\pi,\pi'\in \Theta}P\left(\widehat V_n^{\pi}(x^0)-\widehat V_{n-1}^{\pi'}(x^0)-V^\pi(x^0)+V^{\pi'}(x^0)\geq \frac{\varepsilon}{2}\right) \\
\leq &|\Lambda_n||\Lambda_{n-1}|\sup_{\pi,\pi'\in \Theta}\left(P\left(\widehat V_n^{\pi}(x^0)-V^\pi(x^0)\geq \frac{\varepsilon}{4}\right) + P\left(\widehat V_{n-1}^{\pi'}(x^0)-V^{\pi'}(x^0)\geq \frac{\varepsilon}{4}\right)\right) \\
\leq & 4|\Lambda_n||\Lambda_{k-1}|
 e^{-m_n \frac{\epsilon^2}{8 L^2 M^2}}.
\end{align*}
From the foregoing, we have $\sum_{n=1}^{\infty}P\left(\mathcal{A}_n \cap \mathcal{B}_n\right) < \infty$ since $m_n \rightarrow \infty$ as $n\rightarrow \infty$.  Applying the Borel-Cantelli lemma, we obtain
\begin{align*}
P\left(\mathcal{A}_n \cap \mathcal{B}_n~\mbox{i.o.} \right)=0.
\end{align*}
From the above, it is implied that if $\mathcal{A}_n$ happens infinitely often,
then $\mathcal{B}_n^c$ will also happen infinitely often. Hence,

\begin{align*}
\sum_{n=1}^{\infty}\big[V^{\pi^*_n}(x^0)-V^{\pi^*_{n-1}}(x^0)\big] 
=&\sum_{n:~\mathcal{A}_n \text{occurs}}\big[V^{\pi^*_n}(x^0)-V^{\pi^*_{n-1}}(x^0)\big]+\sum_{n:~\mathcal{A}_n^c \text{occurs}}\big[V^{\pi^*_n}(x^0)-V^{\pi^*_{n-1}}(x^0)\big] \\ 
=&\sum_{n:~\mathcal{A}_n \text{occurs}}\big[V^{\pi^*_n}(x^0)-V^{\pi^*_{n-1}}(x^0)\big]\\ 
=&\sum_{n:~\mathcal{A}_n\cap \mathcal{B}_n \text{occurs}}\big[V^{\pi^*_n}(x^0)-V^{\pi^*_{n-1}}(x^0)\big]\\
&\quad +\sum_{n:~\mathcal{A}_n\cap \mathcal{B}_n^c \text{occurs}}\big[V^{\pi^*_n}(x^0)-V^{\pi^*_{n-1}}(x^0)\big] \\ 
=&\infty \text{~~w.p.1, since $\varepsilon>0$.}
\end{align*}
In the above, the first equality follows from the fact that if the else clause in the second if statement (see \ref{step:3c}) in Algorithm \ref{alg:mras} is hit, then $\pi^*_{n} = \pi^*_{n-1}$.
%
From the last equality above, we conclude that it is a contradiction because, $V^\pi(x^0) > V^{\pi^*}(x^0)$ for any $\pi$ (since $\pi^*$ is the global minimum). The main claim now follows since $\mathcal{A}_n$ can happen only a finite number of times.
\end{proof}

\subsection*{Proof of Theorem \ref{thm:mras}}
\begin{proof}
Once we have established Lemma \ref{lemma:iofo}, the rest of the proof follows in an identical fashion as the proof of Corollary 4.18 of \cite{chang2013simulation}. This is because our algorithm operates in a similar manner as MRAS$_2$ w.r.t. generating the candidate solution using a parameterized family $f(\cdot, \eta)$ and updating the distribution parameter $\eta$. The difference, as mentioned earlier, is the manner in which the samples are generated and the objective (CPT-value) function is estimated. The aforementioned lemma established that the elite sampling and thresholding achieve the same effect as that in MRAS$_2$ and hence the rest of the proof follows from \cite{chang2013simulation}.
\end{proof}

%%%%%%%%%%%%%%%%%%%%%%%%%%%%%%%%%%%%%%%%%%%%%%%%%%%%%%%%%%%%%%%%%%%%%%%%%%%%%%%%
%%%%%%%%%%%%%%%%%%%%%%%%%%%%%%%%%%%%%%%%%%%%%%%%%%%%%%%%%%%%%%%%%%%%%%%%%%%%%%%%

% The estimated errors are presented in the following figure: 



