%!TEX root =  cpt-rl-icml.tex
%For the sake of notational simplicity, we let $X$ denote the r.v. $X^\theta$, i.e., where the parameter $\theta$ is assumed to be fixed for the purpose of CPT-value estimation in this section. \todoc{If we remove the $X^\theta$ business from above, this sentence can be removed.}

%\paragraph{On integrability}
Before diving into the details of CPT value estimation, let us discuss the conditions necessary for the CPT value to be well-defined.
Observe that the first integral in \eqref{eq:cpt-general}, i.e., 
$\int_0^{+\infty} w^+(P(u^+(X)>z)) d z$
may diverge even if the first moment of random variable $u^+(X)$ is finite. 
For example, suppose $U$ has the tail distribution function
$P(U>z)  = \frac{1}{z^2}, z\in [1, +\infty),$
 and $w^+(z)$ takes the form $w(z) = z^{\frac{1}{3}}$. Then, the first integral in \eqref{eq:cpt-general}, i.e.,
$
\int_1^{+\infty} \frac{1}{z^{\frac{2}{3}}} dz
$
does not even exist. A similar argument applies to the second integral in \eqref{eq:cpt-general} as well.

To overcome the above integrability issues, we make different assumptions on the weight and/or utility functions. In particular, we assume that the weight functions $w^+, w^-$ are either 
\begin{inparaenum}[\it (i)]
\item Lipschitz continuous, or
\item \holder continuous, or
\item locally Lipschitz.
\end{inparaenum}
We devise a scheme for estimating \eqref{eq:cpt-general} given only samples from $X$ and show that, under each of the aforementioned assumptions, our estimator (presented next) converges almost surely. 
We also provide sample complexity bounds assuming that the utility functions are bounded.

%%%%%%%%%%%%%%%%%%%%%%%%%%%%%%%%%%%%%%%%%%%%%%%%%%%%%%%%%%%%%%
%%%%%%%%%%%%%%%%%%%%%%%%%%%%%%%%%%%%%%%%%%%%%%%%%%%%%%%%%%%%%%
\subsection{Estimation scheme for \holder continuous weights}
Recall the H\"{o}lder continuity property first in definition 1:
\begin{definition}\label{holder}
{\textbf{\textit{(H\"{o}lder continuity)}}}
If $0 < \alpha \leq 1$, a function $f \in C([a,b])$ is said to satisfy
a H\"{o}lder condition of order $\alpha$ (or to be H\"{o}lder continuous
of order $\alpha$) if $\exists H$, s.t.
\[
\sup_{x \neq y} \frac{| f(x) - f(y) |}{| x-y |^{\alpha}} \leq H .
\]
\end{definition}

In order to ensure integrability of the CPT-value \eqref{eq:cpt-general}, we make the following assumption:\\[1ex]
\textbf{Assumption (A1).}  
The weight functions $w^+, w^-$ are H\"{o}lder continuous with common order $\alpha$. Further,
$\exists \gamma \le \alpha \text{   s.t,  }$ 
$\int_0^{+\infty} P^{\gamma} (u^+(X)>z) dz < +\infty$ and $\int_0^{+\infty} P^{\gamma} (u^-(X)>z) dz < +\infty.$

The above assumption ensures that the CPT-value as defined by \eqref{eq:cpt-general} is finite - see Proposition \ref{prop:Holder-cpt-finite} in 
Appendix \ref{sec:holder-proofs} for a formal proof.


\paragraph{Approximating CPT-value using quantiles:}
Let $\xi^+_{\alpha}$ denote the $\alpha$th quantile of the r.v. $u^+(X)$. Then, it can be seen that (see Proposition \ref{prop:holder-quantile} in Appendix \ref{sec:holder-proofs})
\begin{align}
&\lim_{n \rightarrow \infty} \sum_{i=1}^{n-1} \xi^+_{\frac{i}{n}} \left(w^+\left(\frac{n-i}{n}\right)- w^+\left(\frac{n-i-1}{n}\right) \right) \nonumber\\
&= \int_0^{+\infty} w^+(P(u^+(X)>z)) dz.\label{eq:holder-quant-motiv}
\end{align}
A similar claim holds with $u^-(X)$, $\xi^-_{\alpha}, w^-$ in place of  $u^+(X)$, $\xi^+_{\alpha}, w^+$, respectively. Here $\xi^-_{\alpha}$ denotes the 
$\alpha$th quantile of $u^-(X)$.

However, we do not know the distribution of $u^+(X)$ or $u^-(X)$ and hence, we next present a procedure that uses order statistics for estimating quantiles and this in turn assists estimation of the CPT-value along the lines of \eqref{eq:holder-quant-motiv}. The estimation scheme is presented in Algorithm \ref{alg:holder-est}.

\begin{algorithm}
\caption{CPT-value estimation for \holder continuous weights}
\label{alg:holder-est}
\begin{algorithmic}[1]
\State Simulate $n$ i.i.d. replications follows the distribution of $X$, sort them in ascending order and denoted the ordered sample as 
$X_{[1]}, X_{[2]}, \ldots X_{[n]}$.
%\State Calculate $u^+(X_{[1]}),\ldots u^+(X_{[n]}).$
\State Order the simulated samples and label them as follows: 
$u^+(X_{[1]}),\ldots,u^+(X_{[n]})$.
\State Use $u^+(X_{[i]}), i\in \mathbb{N}\cap (0,n)$ as an approximation for the $\frac{i}{n} th$ quantile of $u^+(X)$, i.e, $\xi_{\frac{i}{n}}, i\in \mathbb{N}\cap (0,n)$.
\State Denote the statistic 
$\overline \C_n^+:=\sum_{i=1}^{n-1} u^+(X_{[i]}) (w^+(\frac{n-i}{n})- w^+(\frac{n-i-1}{n}) )$
\State Repeat the procedure on the sequence $X_{[1]}, X_{[2]}, \ldots X_{[n]}$, with respect to the function $u^-$, 
and denote the statistic $\overline \C_n^-:=\sum_{i=1}^{n-1} u^-(X_{[i]}) (w^-(\frac{n-i}{n})- w^-(\frac{n-i-1}{n}) ) $
\State Return the statistic $\overline \C_n =\overline \C_n^+ - \overline \C_n^-$.
\end{algorithmic}
\end{algorithm}

\subsubsection*{Main results}
%We make the following assumptions on the utility functions:\\[1ex]
\textbf{Assumption (A2).}  The utility functions $u^+(X)$ and $u^-(X)$ are continuous and strictly increasing.

\textbf{Assumption (A2').}  In addition to (A2), the utility functions $u^+(X)$ and $u^-(X)$ are bounded above by $M<\infty$.

For the sample complexity results below, we require (A2'), while (A2) is sufficient to prove asymptotic convergence.

\begin{proposition}(\textbf{Asymptotic convergence.})
\label{prop:holder-asymptotic}
Assume (A1) and also that $F^+(\cdot)$,$F^-(\cdot)$ - the distribution functions of $u^+(X)$, and $u^-(X)$ are Lipschitz continuous with constants $L^+$ and $L^-$, respectively on the interval $(0,+\infty)$, and 
$(-\infty, 0)$ . Then, we have that
\begin{align}
\overline \C_n
\rightarrow
\C(X)
 \text{   a.s. as } n\rightarrow \infty
\end{align}
where $\overline \C_n$ is as defined in Algorithm \ref{alg:holder-est} and $\C(X)$ as in \eqref{eq:cpt-general}.
\end{proposition}
\begin{proof}
See Appendix \ref{sec:holder-proofs}.
\end{proof}
While the above result establishes that $\overline \C_n$ is an unbiased estimator in the asymptotic sense, it is important to know the rate at which the estimate $\overline \C_n$ converges to the CPT-value $\C(X)$. 
The following sample complexity result shows that $O\left(\frac{1}{\epsilon^{2/\alpha}}\right)$ number of samples are required to be $\epsilon$-close to the CPT-value in high probability.
\begin{proposition}(\textbf{Sample complexity.})
\label{prop:holder-dkw}
Assume (A1) and (A2'). Then, $\forall \epsilon >0, \delta >0$, we have
$$
P(\left |\overline \C_n- \C(X) \right| \leq  \epsilon ) > \delta\text{     ,} \forall n \geq \ln(\frac{1}{\delta})\cdot 
\frac{4H^2 M^2}{\epsilon^{2/\alpha}}.$$
\end{proposition}
\begin{proof}
%Notice the the following equivalence:
%$$\sum_{i=1}^{n-1} u^+(X_{[i]}) (w^+(\frac{n-i}{n}) - w^+(\frac{n-i-1}{n})) =  \int_0^M w^+(1-\widehat{F^+_n}(x)) dx, $$
%and also,
%$$\sum_{i=1}^{n-1} u^-(X_{[i]}) (w^-(\frac{n-i}{n}) - w^-(\frac{n-i-1}{n})) =  \int_0^M w^-(1-\widehat{F^-_n}(x)) dx, $$
%
%where $\widehat{F^+_n}(x)$ and $\widehat{F^-_n}(x)$ are the empirical distribution functions (EDFs) of $u^+(X)$
%and $u^-(X)$, defined as follows:
%\begin{align}
%{\widehat F_n}^+(x)=&\frac{1}{n} \sum_{i=1}^n 1_{(u^+(X_i) \leq x)}, 
%{\widehat F_n}^-(x)=\frac{1}{n} \sum_{i=1}^n 1_{(u^-(X_i) \leq x)}.
%\label{eq:edf}
%\end{align}
%The main claim follows from the equivalence mentioned above together with the well-known Dvoretzky-Kiefer-Wolfowitz (DKW) inequality (cf. Chapter 2 of \cite{wasserman2006}).
See Appendix \ref{sec:holder-proofs}.
\end{proof}

\subsubsection{Results for Lipschitz continuous weights}
In the previous section, it was shown that \holder continuous weights incur a sample complexity of order $O\left(\frac1{\epsilon^{2/\alpha}}\right)$ and this is higher than the canonical Monte Carlo rate of $O\left(\frac1{\epsilon^2}\right)$. In this section, we establish that one can achieve the canonical Monte Carlo rate if we consider Lipschitz continuous weights, i.e., the following assumption in place of (A1):
\todoc{How about explaining why we do this? Why do we consider this case? (Every time we do something we should explain why we do it)}
 
\textbf{Assumption (A1').}  The weight functions $w^+, w^-$ are Lipschitz with common constant $L$, and 
$u^+(X)$ and $u^-(X)$ both have bounded first moments.

Setting $\alpha=1$, one can make special cases of the claims regarding asymptotic convergence and sample complexity of Proposition \ref{prop:holder-asymptotic}--\ref{prop:holder-dkw}. However, these results are under  a restrictive Lipschitz assumption on the distribution functions of $u^+(X)$ and $u^-(X)$. Using a different proof technique that employs dominated convergence theorem and DKW inequalities, one can obtain results similar to Proposition \ref{prop:holder-asymptotic}--\ref{prop:holder-dkw} with (A1') and (A2) only. The following claim makes this precise.

\begin{proposition}
\label{prop:lipschitz}
Assume (A1') and (A2). Then, we have that 
$$\overline \C_n
\rightarrow
\C(X)
 \text{   a.s. as } n\rightarrow \infty
$$
In addition, if we assume (A2'), we have $\forall \epsilon >0, \delta >0$ such that
$$
P(\left |\overline \C_n- \C(X) \right| \leq  \epsilon ) > \delta\text{     ,} \forall n \geq \ln(\frac{1}{\delta})\cdot 
\frac{4L^2 M^2}{\epsilon^{2}}.
$$
\end{proposition}
\begin{proof}
See Appendix \ref{sec:lipschitz-proofs}.
\end{proof}

%%%%%%%%%%%%%%%%%%%%%%%%%%%%%%%%%%%%%%%%%%%%%%%%%%%%%%%%%%%%%%
%%%%%%%%%%%%%%%%%%%%%%%%%%%%%%%%%%%%%%%%%%%%%%%%%%%%%%%%%%%%%%
%%%%%%%%%%%%%%%%%%%%%%%%%%%%%%%%%%%%%%%%%%%%%%%%%%%%%%%%%%%%%%
\subsection{Estimation scheme for locally Lipschitz weights and discrete $X$}
<<<<<<< HEAD
\paragraph{Background.}
Here we assume that the r.v. $X$ is discrete valued.
Let $p_i, i=1,\ldots,K$ denote the probability of incurring a gain/loss $x_i, i=1,\ldots,K$. %Assume $x_1 \le \ldots \le x_K$. 
Given a utility function $u$ and weighting function $w$, \textit{\textbf{Prospect theory}} (PT) value is defined as $V(X) = \sum_{i=1}^K u(x_i) w(p_i)$. 
As explained in the introduction, the idea is to take an utility function that is $S$-shaped, so that it satisfies the \textit{diminishing sensitivity}  property. 
If we take the weighting function $w$ to be the identity, then one recovers the classic expected utility. A general weight function inflates low probabilities and deflates high probabilities and this has been shown to be close to the way humans make decisions (see \cite{kahneman1979prospect}, \cite{fennema1997original} for a justification, in particular via empirical tests using human subjects).
However, PT is lacking in some theoretical aspects as it violates first-order \textit{stochastic dominance}.\footnote{Consider the following example from \cite{fennema1997original}: Suppose there are $20$ prospects (outcomes) ranging from $-10$ to $180$, each with probability $0.05$. If the weight function is such that $w(0.05) > 0.05$, then it uniformly overweights all \textit{low-probability} prospects and the resulting PT value is higher than the expected value $85$. This violates stochastic dominance, since a shift in the probability mass from bad outcomes did not result in a better prospect.}

CPT uses a similar measure as PT, except that the weights are a function of cumulative probabilities. First, separate the gains and losses as 
$x_1\le \ldots \le x_l \le 0 \le x_{l+1} \le \ldots \le x_K$. Then, the CPT-value is defined as 
\begin{align}
\label{eq:cpt-discrete}
V(X) = &(u^-(x_1))\cdot w^-(p_1) 
+\sum_{i=2}^l u^-(x_i) \Big(w^-(\sum_{j=1}^i p_j) - w^-(\sum_{j=1}^{i-1} p_j)\Big) 
\\&
 + \sum_{i=l+1}^{K-1} u^+(x_i) \Big(w^+(\sum_{j=i}^K p_j) - w^-(\sum_{j=i+1}^K p_j) \Big)
 + u^+(x_K)\cdot w^+(p_K), 
\end{align} 
where $u^+, u^-$ are utility functions and $w^+, w^-$ are weight functions corresponding to gains and losses, respectively. The utility functions $u^+$ and $u^-$ are non-decreasing, while the weight functions are continuous, non-decreasing and have the range $[0,1]$ with $w^+(0)=w^-(0)=0$ and $w^+(1)=w^-(1)=1$ . 
Unlike PT, the CPT-value does not violate stochastic dominance.\footnote{In the aforementioned example, increasing $w^-(0.05)$ and $w^+(0.05)$ does not impact outcomes other than those on the extreme, i.e., $-10$ and $180$, respectively. For instance, the weight for outcome $100$ would be $w^+(0.45) - w^+(0.40)$. Thus, CPT formalizes the intuitive notion that humans are sensitive to extreme outcomes and relatively insensitive to intermediate ones.}

\paragraph{Estimation scheme.} 
Let $\hat{p_k}= \frac{1}{n} \sum_{i=1}^n I_{\{U =x_k\}}$. Then, we estimate $V(X)$ as follows:
\begin{align}
 \label{eq:cpt-discrete-est}
\hat V_n(X) = 
u^-(x_1)\cdot w^-(\hat p_1)+
& \sum_{i=2}^l u^-(x_i) \Big(w^-(\sum_{j=1}^i \hat p_j) - w^-(\sum_{j=1}^{i-1} \hat p_j)\Big) 
\\
&
+ \sum_{i=l+1}^{K-1} u^+(x_i) \Big(w^+(\sum_{j=i}^K \hat p_j) - w^-(\sum_{j=i+1}^K \hat p_j) \Big)+ u^+(x_K)\cdot w^+(\hat p_K).
\end{align}
Owing to the fact that $\hat{p_k}$ converge a.e to $p_k=P(X_i=x_k)$, with $X_i$ be the sample of $X$ the above estimator obtains strong consistency property according to continuous mapping theorem. 

\paragraph{Sample Complexity}
Before exploring on the convergence speed to the sample estimator, it is necessary to introduce Hoeffding's inequality:
\begin{lemma}
Let $Y_1,...Y_n$ be \emph{independent random variables} satisfying $P(a\leq Y_i \leq b)= 1,$ each i, where $a<b.
$Then for $t>0$,
$$P(\left|\sum_{i=1}^n Y_i -\sum_{i=1}^n E(Y_i)\right| \geq nt ) \leq 2\exp{\{-2nt^2 /(b-a)^2\}} $$
\end{lemma}

\noindent \textbf{Notations:} We will introduce 
\[
F_k = 
\begin{cases}
   \sum_{i=1}^k p_k & \text{if   } k \leq l \\
   \sum_{i=k}^K p_k & \text{if  }  k > l
\end{cases}  
\]

and $\hat F_k$ retains the same form as $F_k$ by replacing $p_k$ by $\hat p_k$ 
\\

\noindent The Hoeffding inequality suggests the following proposition:
\begin{proposition}
Let $F_k$ and $\hat F_k$ as introduced above, Then for every $\epsilon >0$, 
$$P(|\hat{F_k}-F_k| > \epsilon) \leq 2 e^{-2n \epsilon^2} $$
\end{proposition}
\begin{proof}
We focus on the case when $k > l$,and the case of $k \leq l$ will be proved through the same fashion.
Notice that when $k>l$,  $\hat F_k =I_{(U_i \geq  x_k) }$ and the random variables are independent to each other for each i, and it is bounded by 1. 
The probability $P(\left|F_k- F_k \right| > \epsilon)$ is equal to 
\begin{align*}
&
P(\left|\hat{F_k}- F_k \right| > \epsilon) \\ & = P(\left| \frac{1}{n} \sum_{i=1}^n I_{\{U_i \geq
x_k\}} - \frac{1}{n} \sum_{i=1}^n E(I_{\{U_i \geq x_k\}}) \right| > \epsilon) \\ & = P(\left|
\sum_{i=1}^n I_{\{U_i \geq x_k\}} - \sum_{i=1}^n E(I_{\{U_i \geq x_k\}}) \right| > n\epsilon) \\ &
    \leq 2e^{-2n \epsilon^2}
\end{align*}

\end{proof}
Proposition 5 gives a convergence rate of $\hat{F_k}$ to the value $F_k$, regardless of what k is. 
Additionally, since $w^+$ and $w^-$ are both locally Lipschitz as indicated in the paper, we can explore the sample complexity of the estimation algorithm through the following lemma: 

\begin{theorem}[Sample Complexity: discrete case]
\label{thm:sample-complexity}
Denote $L=\max\{L_k, k=2...K\} $, where $L_k$ is the local Lipschitz constant of function $w^-(x)$ at points
$F_k$, where $k=1,...l$, and of function $w^+(x)$ at points $k=l+1,...K$. 
And let $A=\max\{x_k, k=1...K\}$, $\delta =\min\{\delta_k\}$, where $\delta_k$ is the half length of the interval centered at point $F_k$ where locally Lipschitz property with constant $L_k$ holds.
For any $\epsilon$, $\rho$,let $M=\min(\delta^2, \epsilon^2/(KLA)^2)$, and we have 
\begin{align}
P(\left|
\hat V_n(X) -V(X)
\right| \leq \epsilon) > 1-\rho \text{        ,} \forall n> \frac{\ln(\frac{4K}{a})} { M} 
\end{align}

\end{theorem}

Before proving the preceding theorem, we will introduce the following proposition : 
\begin{proposition}
Following the same notations and conditions introduced in theorem 2, and assume that
$w$ is Locally Lipschitz continuous with constants $L_1,....L_K$ on the points $F_1,....F_K$ as written in the statement of theorem 2, we have
$$P(\left| \sum_{i=1}^K x_k w(\hat{F_k}) - \sum_{i=1}^K x_k w(F_k) \right| >\epsilon) < K\cdot (
e^{-\delta^2\cdot 2n} + e^{-\epsilon^2 2n/(KLA)^2}) $$ 
\end{proposition}

\begin{proof}
Observe that

\begin{align*}
&
P(\left| \sum_{k=1}^K x_k w(\hat{F_k}) - \sum_{k=1}^K x_k w(F_k) \right| >\epsilon) \\ & = P (
\bigcup_{k=1}^K \left| x_k w(\hat{F_k}) -x_k w(F_k) \right| > \frac {\epsilon} {K}) \\ & \leq
    \sum_{k=1}^K P (\left| x_k w(\hat{F_k}) -x_k w(F_k) \right| > \frac {\epsilon} {K})
\end{align*}
Notice that $\forall k =1,....K$
$[{p_k}- \delta, {p_k}+\delta)$,
the function $w$ is locally Lipschitz with common constant $L$.
Therefore, for each k, we can decompose the probability as 
\begin{align*}
& P (\left| x_k w(\hat{F_k}) -x_k w(F_k) \right| > \frac {\epsilon} {K}) \\ & = P ( [ \left| F_k -
\hat{F_k} \right| >\delta ] \bigcap [ \left| x_k w(\hat{F_k}) -x_k w(F_k) \right| ] > \frac
{\epsilon} {K}) + P ( [ \left| F_k - \hat{F_k} \right| \leq\delta ] \bigcap [ \left| x_k
    w(\hat{F_k}) -x_k w(F_k) \right| ] > \frac {\epsilon} {K}) \\ & \leq P ( \left| F_k - \hat{F_k}
    \right| >\delta) + P ( [ \left| F_k - \hat{F_k} \right| \leq\delta ] \bigcap [ \left| x_k
    w(\hat{F_k}) -x_k w(F_k) \right| ] > \frac {\epsilon} {K})
\end{align*}
 
According to the property of locally Lipschitz continuous,
we have
\begin{align*}
& P ( [ \left| F_k - \hat{F_k} \right| \leq\delta ] \bigcap [ \left| x_k w(\hat{F_k}) -x_k w(F_k)
\right| ] > \frac {\epsilon} {K}) \\ & \leq P(x_k L \left| F_k - \hat{F_k} \right| > \frac
    {\epsilon} {K}) \leq e^ {-\epsilon\cdot 2n /(K L x_k)^2} \leq e^ {-\epsilon\cdot 2n /(K L A)^2}
    \text{     for    } \forall k
\end{align*}
And similarly,
\begin{align*}
& P(\left| F_k - \hat{F_k} \right| > \delta) \\ & \leq e^{-\delta^2 /2n} \text{    for     } \forall
    n
\end{align*}
And as a result,
\begin{align*}
& P(\left| \sum_{k=1}^K x_k w(\hat{F_k}) - \sum_{k=1}^K x_k w(F_k) \right| >\epsilon) \\ & \leq
\sum_{k=1}^K P (\left| x_k w(\hat{F_k}) -x_k w(F_k) \right| > \frac {\epsilon} {K}) \\ & \leq
             \sum_{k=1}^K e^{-\delta^2\cdot 2n} + e^{-\epsilon^2 \cdot 2n/ (KLA)^2} \\ & =K\cdot
    (e^{-\delta^2\cdot 2n} + e^{-\epsilon^2 \cdot 2n/ (KLA)^2})
\end{align*}

\end{proof}

By giving the above proposition, we can prove theorem 2:
\begin{proof}[Proof of theorem 2:]
Since the functions $w^-$ and $w^+$ all locally Lipschitz in the according points and with the according constants introduced in theorem 2, it is suggestive only to write $w$ uniformly in place of $w^-$ and $w^+$ in the separate cases $1\leq k\leq l$ and $k> l$, in order to avoid unnecessary technicalities. 
The proof is equivalently to show that
\begin{align}
P(\left|\sum_{i=1}^K u(x_k) \cdot(w(\hat{F_k})- w(\hat F_{k+1}) )
-  
\sum_{i=1}^K u(x_k) \cdot(w(F_k)- w(F_{k+1}) )
\right| \leq \epsilon) > 1-\rho
\text{      ,     } \forall n> \frac{\ln(\frac{4K}{a})} { M} 
\end{align}
under which $w$ is Locally Lipschitz continuous with constants $L_1,....L_K$ on the points $F_1,....F_K$ as written in the statement of theorem 2.
Observe that by repeating the identical procedure in the proof of proposition 6 one can show that
$$P(\left| \sum_{i=1}^K x_k w(\hat F_{k+1}) - \sum_{i=1}^K x_k w(F_{k+1}) \right| >\epsilon) <
K\cdot ( e^{-\delta^2\cdot 2n} + e^{-\epsilon^2 2n/(KLA)^2})
$$
Therefore,
\begin{align*}
& P(\left|\sum_{i=1}^K x_k \cdot(w(\hat{F_k})- w(\hat F_{k+1}) ) -  \sum_{i=1}^K x_k \cdot(w(F_k)-
w(F_{k+1}) ) \right| > \epsilon) \\ & \leq P(\left|\sum_{i=1}^K x_k \cdot(w(\hat{F_k})) -
    \sum_{i=1}^K x_k \cdot(w(F_k)) \right| > \epsilon/2) + P(\left|\sum_{i=1}^K x_k
    \cdot(w(\hat F_{k+1})) -  \sum_{i=1}^K x_k \cdot(w(F_{k+1})) \right| > \epsilon/2) \\ & \leq 2K
    (e^{-\delta^2\cdot 2n} + e^{-\epsilon^2 2n/(KLA)^2})
\end{align*}
And by introducing the notation $M=\min(\delta^2, \epsilon^2/(KLA)^2)$, one can conclude the sample complexity property stated in the theorem.




\end{proof}

=======
\paragraph{Background.}
Here we assume that the r.v. $X$ is discrete valued.
Let $p_i, i=1,\ldots,K$ denote the probability of incurring a gain/loss $x_i, i=1,\ldots,K$. %Assume $x_1 \le \ldots \le x_K$. 
Given a utility function $u$ and weighting function $w$, \textit{\textbf{Prospect theory}} (PT) value is defined as $V(X) = \sum_{i=1}^K u(x_i) w(p_i)$. 
As explained in the introduction, the idea is to take an utility function that is $S$-shaped, so that it satisfies the \textit{diminishing sensitivity}  property. 
If we take the weighting function $w$ to be the identity, then one recovers the classic expected utility. A general weight function inflates low probabilities and deflates high probabilities and this has been shown to be close to the way humans make decisions (see \cite{kahneman1979prospect}, \cite{fennema1997original} for a justification, in particular via empirical tests using human subjects).
However, PT is lacking in some theoretical aspects as it violates first-order \textit{stochastic dominance}.\footnote{Consider the following example from \cite{fennema1997original}: Suppose there are $20$ prospects (outcomes) ranging from $-10$ to $180$, each with probability $0.05$. If the weight function is such that $w(0.05) > 0.05$, then it uniformly overweights all \textit{low-probability} prospects and the resulting PT value is higher than the expected value $85$. This violates stochastic dominance, since a shift in the probability mass from bad outcomes did not result in a better prospect.}

CPT uses a similar measure as PT, except that the weights are a function of cumulative probabilities. First, separate the gains and losses as 
$x_1\le \ldots \le x_l \le 0 \le x_{l+1} \le \ldots \le x_K$. Then, the CPT-value is defined as 
\begin{align}
\label{eq:cpt-discrete}
V(X) = & \sum_{i=l+1}^{K-1} u^+(x_i) \Big(w^+(\sum_{j=i}^K p_j) - w^+(\sum_{j=i+1}^K p_j) \Big)
- \sum_{i=1}^{l} u^-(x_i) \Big(w^-(\sum_{j=i}^l p_j) - w^-(\sum_{j=i+1}^l p_j) \Big), 
\end{align} 
where $u^+, u^-$ are utility functions and $w^+, w^-$ are weight functions corresponding to gains and losses, respectively. The utility functions $u^+$ and $u^-$ are non-decreasing, while the weight functions are continuous, non-decreasing and have the range $[0,1]$ with $w^+(0)=w^-(0)=0$ and $w^+(1)=w^-(1)=1$ . 
Unlike PT, the CPT-value does not violate stochastic dominance.\footnote{In the aforementioned example, increasing $w^-(0.05)$ and $w^+(0.05)$ does not impact outcomes other than those on the extreme, i.e., $-10$ and $180$, respectively. For instance, the weight for outcome $100$ would be $w^+(0.45) - w^+(0.40)$. Thus, CPT formalizes the intuitive notion that humans are sensitive to extreme outcomes and relatively insensitive to intermediate ones.}

\paragraph{Estimation scheme.} 
Let $\hat{p_k}= \frac{1}{n} \sum_{i=1}^n I_{\{X =x_k\}}$. Then, we estimate $V(X)$ as follows:
\begin{align}
 \label{eq:cpt-discrete-est}
\hat V(X) = & \sum_{i=l+1}^{K-1} u^+(x_i) \Big(w^+(\sum_{j=i}^K \hat p_j) - w^+(\sum_{j=i+1}^K \hat p_j) \Big)
- \sum_{i=1}^{l} u^-(x_i) \Big(w^-(\sum_{j=i}^l \hat p_j) - w^-(\sum_{j=i+1}^l \hat p_j) \Big).
\end{align}

\subsubsection*{Main result}
The following proposition presents a sample complexity result for the discrete valued $X$ under the following assumption:\\
\textbf{Assumption (A3).}  The weight functions $w^+(X)$ and $w^-(X)$ are locally Lipschitz continuous, i.e., for any $x$, there exists a $\delta>0$, such that
$$| w^+(x) - w^+(y) | \leq L_x |x-y|, \text{ for all } y \in (x-\delta,x+\delta) $$.\\

We denote $L=\max\{L_k, k=2...K\}$,  where $L_k$ is the Lipschitz constant at $F_k = \sum_{i=k}^K p_i$.

\todoj[inline]{Fix claim to be for $\hat V$}
\begin{proposition}
$A=\max\{x_k, k=1...K\}$, $\delta =\min\{\delta_k\}$, where $\delta_k$ is the half length of the interval that locally Lipschitz of the point $F_k$ holds.
Suppose $\hat{F_k}$ is the empirical estimation of $F_k$, then 
$$P(\left| \sum_{i=1}^K x_k w(\hat{F_k}) - \sum_{i=1}^K x_k w(F_k) \right| >\epsilon) < K\cdot (
e^{-\delta^2\cdot 2n} + e^{-\epsilon^2 2n/(KLA)^2}) $$ 
\end{proposition}
\begin{proof}
 See Section \ref{sec:proofs-discrete}.
\end{proof}

>>>>>>> origin/master

