Here we assume that the r.v. $X$ is discrete valued. 
Let $p_i, i=1,\ldots,K$ denote the probability of incurring a gain/loss $x_i, i=1,\ldots,K$, where 
$x_1\le \ldots \le x_l \le 0 \le x_{l+1} \le \ldots \le x_K$ and  let
\begin{align}
\label{eq:Fk}
 F_k = 
\begin{cases}
   \sum_{i=1}^k p_k  \text{ if   } k \leq l \text{ and }
   \sum_{i=k}^K p_k  \text{ if  }  k > l.
\end{cases}  
\end{align}
Then, the CPT-value is defined as 
\begin{small}
\begin{align*}
&\C(X) \!=\! (u^-(x_1))\cdot w^-(p_1) 
\!+\!\sum_{i=2}^l u^-(x_i) \Big(w^-(F_i) - w^-(F_{i-1})\Big) \\
& + \sum_{i=l+1}^{K-1} u^+(x_i) \Big(w^+(F_i) - w^+(F_{i+1}) \Big)
 + u^+(x_K)\cdot w^+(p_K),
\end{align*} 
\end{small}
where $u^+, u^-$ are utility functions and $w^+, w^-$ are weight functions corresponding to gains and losses, respectively. The utility functions $u^+$ and $u^-$ are non-decreasing, while the weight functions are continuous, non-decreasing and have the range $[0,1]$ with $w^+(0)=w^-(0)=0$ and $w^+(1)=w^-(1)=1$. 

\paragraph{Estimation scheme.} 
Let $\hat p_k= \frac{1}{n} \sum_{i=1}^n I_{\{U =x_k\}}$ and 
\begin{align}
\label{eq:Fkhat}
 \hat F_k = 
\begin{cases}
   \sum_{i=1}^k \hat p_k  \text{ if   } k \leq l \text{ and }
   \sum_{i=k}^K \hat p_k  \text{ if  }  k > l.
\end{cases}  
\end{align}
Then, we estimate $\C(X)$ as follows:
\begin{small}
\begin{align}
&\overline \C_n \!=\! 
u^-(x_1)\cdot w^-(\hat p_1) \!+\!\sum_{i=2}^l u^-(x_i) \Big(w^-(\hat F_i) - w^-( \hat F_{i-1})\Big) 
\nonumber\\
&
+ \sum_{i=l+1}^{K-1} u^+(x_i) \Big(w^+(\hat F_i) - w^+(\hat F_{i+1}) \Big) + u^+(x_K)\cdot w^+(\hat p_K). \label{eq:cpt-discrete-est}
\end{align}
\end{small}
%Because $\hat{p_k}$ converge a.e to $p_k=P(X_i=x_k)$, with $X_i$ be the ith sample of $X$, the above estimator is  strong consistent property by the continuous mapping theorem. 
%The following proposition presents a sample complexity result for the discrete-valued $X$ under the following assumption:\\
\textbf{Assumption (A3).}  The weight functions $w^+(X)$ and $w^-(X)$ are locally Lipschitz continuous, i.e., for any $x$, there exist  $L< \infty$ and $\delta>0$, such that
$$| w^+(x) - w^+(y) | \leq L_x |x-y|, \text{ for all } y \in (x-\delta,x+\delta). $$
The main result for discrete-valued $X$ is given below.
\begin{proposition}
\label{prop:sample-complexity-discrete}
Assume (A3). Let $L=\max\{L_k, k=2...K\} $, where $L_k$ is the local Lipschitz constant of function $w^-(x)$ at points
$F_k$, where $k=1,...l$, and of function $w^+(x)$ at points $k=l+1,...K$. 
Let $A=\max\{x_k, k=1...K\}$, $\delta =\min\{\delta_k\}$, where $\delta_k$ is the half the length of the interval centered at point $F_k$ where the locally Lipschitz property with constant $L_k$ holds.
For any $\epsilon,\rho >0$, we have 
\begin{align}
P(\left|
\overline \C_n -\C(X)
\right| \leq \epsilon) > 1-\rho, \forall n> \frac{\ln(\frac{4K}{a})} { M}, 
\end{align}
where $M=\min(\delta^2, \epsilon^2/(KLA)^2)$.
\end{proposition}
In comparison to Propositions \ref{prop:holder-dkw} and \ref{prop:lipschitz}, 
observe that the sample complexity for discrete $X$ scales with the local Lipschitz constant $L$ and this can be much smaller the global Lipschitz constant of the weight functions or the weight functions may not be Lipschitz globally.  
\begin{proof}
 See Section \ref{sec:proofs-discrete}.
\end{proof}

